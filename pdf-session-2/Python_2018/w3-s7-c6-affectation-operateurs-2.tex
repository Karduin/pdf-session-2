    \hypertarget{les-instructions-et-autres-revisituxe9es}{%
\section{\texorpdfstring{Les instructions \texttt{+=} et autres
revisitées}{Les instructions += et autres revisitées}}\label{les-instructions-et-autres-revisituxe9es}}

    \hypertarget{compluxe9ment---niveau-intermuxe9diaire}{%
\subsection{Complément - niveau
intermédiaire}\label{compluxe9ment---niveau-intermuxe9diaire}}

    Nous avons vu en première semaine (Séquence ``Les types numériques'')
une première introduction à l'instruction \texttt{+=} et ses dérivées
comme \texttt{*=}, \texttt{**=}, etc.

    \hypertarget{ces-constructions-ont-une-duxe9finition-uxe0-guxe9omuxe9trie-variable}{%
\subsubsection{Ces constructions ont une définition à géométrie
variable}\label{ces-constructions-ont-une-duxe9finition-uxe0-guxe9omuxe9trie-variable}}

    En C quand on utilise \texttt{+=} (ou encore \texttt{++}) on modifie la
mémoire en place - historiquement, cet opérateur permettait au
programmeur d'aider à l'optimisation du code pour utiliser les
instructions assembleur idoines.\\

Ces constructions en Python s'inspirent clairement de C, aussi dans
l'esprit ces constructions devraient fonctionner en \textbf{modifiant}
l'objet référencé par la variable.\\

Mais les types numériques en Python ne sont \textbf{pas mutables}, alors
que les listes le sont. Du coup le comportement de \texttt{+=} est
\textbf{différent} selon qu'on l'utilise sur un nombre ou sur une liste,
ou plus généralement selon qu'on l'invoque sur un type mutable ou non.
Voyons cela sur des exemples très simples~:

    \begin{Verbatim}[commandchars=\\\{\}]
{\color{incolor}In [{\color{incolor}1}]:} \PY{c+c1}{\PYZsh{} Premier exemple avec un entier}
        
        \PY{c+c1}{\PYZsh{} on commence avec une référence partagée}
        \PY{n}{a} \PY{o}{=} \PY{n}{b} \PY{o}{=} \PY{l+m+mi}{3}
        \PY{n}{a} \PY{o+ow}{is} \PY{n}{b}
\end{Verbatim}


\begin{Verbatim}[commandchars=\\\{\}]
{\color{outcolor}Out[{\color{outcolor}1}]:} True
\end{Verbatim}
            
    \begin{Verbatim}[commandchars=\\\{\}]
{\color{incolor}In [{\color{incolor}2}]:} \PY{c+c1}{\PYZsh{} on utilise += sur une des deux variables}
        \PY{n}{a} \PY{o}{+}\PY{o}{=} \PY{l+m+mi}{1}
        
        \PY{c+c1}{\PYZsh{} ceci n\PYZsq{}a pas modifié b}
        \PY{c+c1}{\PYZsh{} c\PYZsq{}est normal, l\PYZsq{}entier n\PYZsq{}est pas mutable}
        
        \PY{n+nb}{print}\PY{p}{(}\PY{n}{a}\PY{p}{)}
        \PY{n+nb}{print}\PY{p}{(}\PY{n}{b}\PY{p}{)}
        \PY{n+nb}{print}\PY{p}{(}\PY{n}{a} \PY{o+ow}{is} \PY{n}{b}\PY{p}{)}
\end{Verbatim}


    \begin{Verbatim}[commandchars=\\\{\}]
4
3
False

    \end{Verbatim}

    \begin{Verbatim}[commandchars=\\\{\}]
{\color{incolor}In [{\color{incolor}3}]:} \PY{c+c1}{\PYZsh{} Deuxième exemple, cette fois avec une liste}
        
        \PY{c+c1}{\PYZsh{} la même référence partagée}
        \PY{n}{a} \PY{o}{=} \PY{n}{b} \PY{o}{=} \PY{p}{[}\PY{p}{]}
        \PY{n}{a} \PY{o+ow}{is} \PY{n}{b}
\end{Verbatim}


\begin{Verbatim}[commandchars=\\\{\}]
{\color{outcolor}Out[{\color{outcolor}3}]:} True
\end{Verbatim}
            
    \begin{Verbatim}[commandchars=\\\{\}]
{\color{incolor}In [{\color{incolor}4}]:} \PY{c+c1}{\PYZsh{} pareil, on fait += sur une des variables}
        \PY{n}{a} \PY{o}{+}\PY{o}{=} \PY{p}{[}\PY{l+m+mi}{1}\PY{p}{]}
        
        \PY{c+c1}{\PYZsh{} cette fois on a modifié a et b}
        \PY{c+c1}{\PYZsh{} car += a pu modifier la liste en place}
        \PY{n+nb}{print}\PY{p}{(}\PY{n}{a}\PY{p}{)}
        \PY{n+nb}{print}\PY{p}{(}\PY{n}{b}\PY{p}{)}
        \PY{n+nb}{print}\PY{p}{(}\PY{n}{a} \PY{o+ow}{is} \PY{n}{b}\PY{p}{)}
\end{Verbatim}


    \begin{Verbatim}[commandchars=\\\{\}]
[1]
[1]
True

    \end{Verbatim}

    Vous voyez donc que la sémantique de \texttt{+=} (c'est bien entendu le
cas pour toutes les autres formes d'instructions qui combinent
l'affectation avec un opérateur) \textbf{est différente} suivant que
l'objet référencé par le terme de gauche est \textbf{mutable ou
immuable}.\\

    Pour cette raison, c'est là une opinion personnelle, cette famille
d'instructions n'est pas le trait le plus réussi dans le langage, et je
ne recommande pas de l'utiliser.

    \hypertarget{pruxe9cision-sur-la-duxe9finition-de}{%
\subsubsection{\texorpdfstring{Précision sur la définition de
\texttt{+=}}{Précision sur la définition de +=}}\label{pruxe9cision-sur-la-duxe9finition-de}}

    Nous avions dit en première semaine, et en première approximation, que~:

\begin{Shaded}
\begin{Highlighting}[]
\NormalTok{x }\OperatorTok{+=}\NormalTok{ y}
\end{Highlighting}
\end{Shaded}

était équivalent à~:

\begin{Shaded}
\begin{Highlighting}[]
\NormalTok{x }\OperatorTok{=}\NormalTok{ x }\OperatorTok{+}\NormalTok{ y}
\end{Highlighting}
\end{Shaded}

    Au vu de ce qui précède, on voit que ce n'est \textbf{pas tout à fait
exact}, puisque~:

    \begin{Verbatim}[commandchars=\\\{\}]
{\color{incolor}In [{\color{incolor}5}]:} \PY{c+c1}{\PYZsh{} si on fait x += y sur une liste}
        \PY{c+c1}{\PYZsh{} on fait un effet de bord sur la liste}
        \PY{c+c1}{\PYZsh{} comme on vient de le voir}
        
        \PY{n}{a} \PY{o}{=} \PY{p}{[}\PY{p}{]}
        \PY{n+nb}{print}\PY{p}{(}\PY{l+s+s2}{\PYZdq{}}\PY{l+s+s2}{avant}\PY{l+s+s2}{\PYZdq{}}\PY{p}{,} \PY{n+nb}{id}\PY{p}{(}\PY{n}{a}\PY{p}{)}\PY{p}{)}
        \PY{n}{a} \PY{o}{+}\PY{o}{=} \PY{p}{[}\PY{l+m+mi}{1}\PY{p}{]}
        \PY{n+nb}{print}\PY{p}{(}\PY{l+s+s2}{\PYZdq{}}\PY{l+s+s2}{après}\PY{l+s+s2}{\PYZdq{}}\PY{p}{,} \PY{n+nb}{id}\PY{p}{(}\PY{n}{a}\PY{p}{)}\PY{p}{)}
\end{Verbatim}


    \begin{Verbatim}[commandchars=\\\{\}]
avant 82877176
après 82877176

    \end{Verbatim}

    \begin{Verbatim}[commandchars=\\\{\}]
{\color{incolor}In [{\color{incolor}6}]:} \PY{c+c1}{\PYZsh{} alors que si on fait x = x + y sur une liste}
        \PY{c+c1}{\PYZsh{} on crée un nouvel objet liste}
        
        \PY{n}{a} \PY{o}{=} \PY{p}{[}\PY{p}{]}
        \PY{n+nb}{print}\PY{p}{(}\PY{l+s+s2}{\PYZdq{}}\PY{l+s+s2}{avant}\PY{l+s+s2}{\PYZdq{}}\PY{p}{,} \PY{n+nb}{id}\PY{p}{(}\PY{n}{a}\PY{p}{)}\PY{p}{)}
        \PY{n}{a} \PY{o}{=} \PY{n}{a} \PY{o}{+} \PY{p}{[}\PY{l+m+mi}{1}\PY{p}{]}
        \PY{n+nb}{print}\PY{p}{(}\PY{l+s+s2}{\PYZdq{}}\PY{l+s+s2}{après}\PY{l+s+s2}{\PYZdq{}}\PY{p}{,} \PY{n+nb}{id}\PY{p}{(}\PY{n}{a}\PY{p}{)}\PY{p}{)}
\end{Verbatim}


    \begin{Verbatim}[commandchars=\\\{\}]
avant 82981752
après 82877496

    \end{Verbatim}

    Vous voyez donc que vis-à-vis des références partagées, ces deux façons
de faire mènent à un résultat différent.