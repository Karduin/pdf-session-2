    \hypertarget{la-boucle-while-else}{%
\section{\texorpdfstring{La boucle \texttt{while} \ldots{}
\texttt{else}}{La boucle while \ldots{} else}}\label{la-boucle-while-else}}

    \hypertarget{compluxe9ment---niveau-basique}{%
\subsection{Complément - niveau
basique}\label{compluxe9ment---niveau-basique}}

    \hypertarget{boucles-sans-fin---break}{%
\subsubsection{\texorpdfstring{Boucles sans fin -
\texttt{break}}{Boucles sans fin - break}}\label{boucles-sans-fin---break}}

    Utiliser \texttt{while} plutôt que \texttt{for} est une affaire de style
et d'habitude. Cela dit en Python, avec les notions d'itérable et
d'itérateur, on a tendance à privilégier l'usage du \texttt{for} pour
les boucles finies et déterministes.\\

    Le \texttt{while} reste malgré tout d'un usage courant, et notamment
avec une condition \texttt{True}.

Par exemple le code de l'interpréteur interactif de python pourrait
ressembler, vu de très loin, à quelque chose comme ceci~:

\begin{verbatim}
while True:
    print(eval(read()))
\end{verbatim}

    Notez bien par ailleurs que les instructions \texttt{break} et
\texttt{continue} fonctionnent, à l'intérieur d'une boucle
\texttt{while}, exactement comme dans un \texttt{for}, c'est-à-dire
que~:

\begin{itemize}
	\item 
	\texttt{continue} termine l'itération courante mais reste dans
	la boucle, alors que
	\item
	\texttt{break} interrompt l'itération courante et
	sort également de la boucle.
\end{itemize}

    \hypertarget{compluxe9ment---niveau-intermuxe9diaire}{%
\subsection{Complément - niveau
intermédiaire}\label{compluxe9ment---niveau-intermuxe9diaire}}

    \hypertarget{rappel-sur-les-conditions}{%
\subsubsection{Rappel sur les
conditions}\label{rappel-sur-les-conditions}}

    On peut utiliser dans une boucle \texttt{while} toutes les formes de
conditions que l'on a vues à l'occasion de l'instruction \texttt{if}.\\

Dans le contexte de la boucle \texttt{while} on comprend mieux,
toutefois, pourquoi le langage autorise d'écrire des conditions dont le
résultat n'est \textbf{pas nécessairement un booléen}. Voyons cela sur
un exemple simple~:

    \begin{Verbatim}[commandchars=\\\{\}]
{\color{incolor}In [{\color{incolor}1}]:} \PY{c+c1}{\PYZsh{} une autre façon de parcourir une liste}
        \PY{n}{liste} \PY{o}{=} \PY{p}{[}\PY{l+s+s1}{\PYZsq{}}\PY{l+s+s1}{a}\PY{l+s+s1}{\PYZsq{}}\PY{p}{,} \PY{l+s+s1}{\PYZsq{}}\PY{l+s+s1}{b}\PY{l+s+s1}{\PYZsq{}}\PY{p}{,} \PY{l+s+s1}{\PYZsq{}}\PY{l+s+s1}{c}\PY{l+s+s1}{\PYZsq{}}\PY{p}{]}
        
        \PY{k}{while} \PY{n}{liste}\PY{p}{:}
            \PY{n}{element} \PY{o}{=} \PY{n}{liste}\PY{o}{.}\PY{n}{pop}\PY{p}{(}\PY{p}{)}
            \PY{n+nb}{print}\PY{p}{(}\PY{n}{element}\PY{p}{)}
\end{Verbatim}


    \begin{Verbatim}[commandchars=\\\{\}]
c
b
a

    \end{Verbatim}

    \hypertarget{une-curiosituxe9-la-clause-else}{%
\subsubsection{\texorpdfstring{Une curiosité~: la clause
\texttt{else}}{Une curiosité~: la clause else}}\label{une-curiosituxe9-la-clause-else}}

    Signalons enfin que la boucle \texttt{while} - au même titre d'ailleurs
que la boucle \texttt{for}, peut être assortie
\href{https://docs.python.org/3/reference/compound_stmts.html\#the-while-statement}{d'une
clause \texttt{else}}, qui est exécutée à la fin de la boucle,
\textbf{sauf dans le cas d'une sortie avec \texttt{break}}

    \begin{Verbatim}[commandchars=\\\{\}]
{\color{incolor}In [{\color{incolor}2}]:} \PY{c+c1}{\PYZsh{} Un exemple de while avec une clause else}
        
        \PY{c+c1}{\PYZsh{} si break\PYZus{}mode est vrai on va faire un break}
        \PY{c+c1}{\PYZsh{} après le premier élément de la liste}
        \PY{k}{def} \PY{n+nf}{scan}\PY{p}{(}\PY{n}{liste}\PY{p}{,} \PY{n}{break\PYZus{}mode}\PY{p}{)}\PY{p}{:}
        
            \PY{c+c1}{\PYZsh{} un message qui soit un peu parlant}
            \PY{n}{message} \PY{o}{=} \PY{l+s+s2}{\PYZdq{}}\PY{l+s+s2}{avec break}\PY{l+s+s2}{\PYZdq{}} \PY{k}{if} \PY{n}{break\PYZus{}mode} \PY{k}{else} \PY{l+s+s2}{\PYZdq{}}\PY{l+s+s2}{sans break}\PY{l+s+s2}{\PYZdq{}}
            \PY{n+nb}{print}\PY{p}{(}\PY{n}{message}\PY{p}{)}
            \PY{k}{while} \PY{n}{liste}\PY{p}{:}
                \PY{n+nb}{print}\PY{p}{(}\PY{n}{liste}\PY{o}{.}\PY{n}{pop}\PY{p}{(}\PY{p}{)}\PY{p}{)}
                \PY{k}{if} \PY{n}{break\PYZus{}mode}\PY{p}{:}
                    \PY{k}{break}
            \PY{k}{else}\PY{p}{:}
                \PY{n+nb}{print}\PY{p}{(}\PY{l+s+s1}{\PYZsq{}}\PY{l+s+s1}{else...}\PY{l+s+s1}{\PYZsq{}}\PY{p}{)}
\end{Verbatim}


    \begin{Verbatim}[commandchars=\\\{\}]
{\color{incolor}In [{\color{incolor}3}]:} \PY{c+c1}{\PYZsh{} sortie de la boucle sans break}
        \PY{c+c1}{\PYZsh{} on passe par else}
        \PY{n}{scan}\PY{p}{(}\PY{p}{[}\PY{l+s+s1}{\PYZsq{}}\PY{l+s+s1}{a}\PY{l+s+s1}{\PYZsq{}}\PY{p}{]}\PY{p}{,} \PY{k+kc}{False}\PY{p}{)}
\end{Verbatim}


    \begin{Verbatim}[commandchars=\\\{\}]
sans break
a
else{\ldots}

    \end{Verbatim}

    \begin{Verbatim}[commandchars=\\\{\}]
{\color{incolor}In [{\color{incolor}4}]:} \PY{c+c1}{\PYZsh{} on sort de la boucle par le break}
        \PY{n}{scan}\PY{p}{(}\PY{p}{[}\PY{l+s+s1}{\PYZsq{}}\PY{l+s+s1}{a}\PY{l+s+s1}{\PYZsq{}}\PY{p}{]}\PY{p}{,} \PY{k+kc}{True}\PY{p}{)}
\end{Verbatim}


    \begin{Verbatim}[commandchars=\\\{\}]
avec break
a

    \end{Verbatim}

    Ce trait est toutefois \textbf{très rarement} utilisé.