    \hypertarget{la-ligne-shebang}{%
\section{\texorpdfstring{La ligne
\emph{shebang}}{La ligne shebang}}\label{la-ligne-shebang}}

    \begin{verbatim}
#!/usr/bin/env python3
\end{verbatim}

    \hypertarget{compluxe9ment---niveau-avancuxe9}{%
\subsection{Complément - niveau
avancé}\label{compluxe9ment---niveau-avancuxe9}}

    Ce complément est uniquement valable pour macOS et Linux.

    \hypertarget{le-besoin}{%
\subsubsection{Le besoin}\label{le-besoin}}

    Nous avons vu dans la vidéo que, pour lancer un programme Python, on
fait depuis le terminal~:

    \begin{verbatim}
$ python3 mon_module.py
\end{verbatim}

    Lorsqu'il s'agit d'un programme que l'on utilise fréquemment, on n'est
pas forcément dans le répertoire où se trouve le programme Python.
Aussi, dans ce cas, on peut utiliser un chemin ``absolu'', c'est-à-dire
à partir de la racine des noms de fichiers, comme par exemple~:

    \begin{verbatim}
$ python3 /le/chemin/jusqu/a/mon_module.py
\end{verbatim}

    Sauf que c'est assez malcommode, et cela devient vite pénible à la
longue.

    \hypertarget{la-solution}{%
\subsubsection{La solution}\label{la-solution}}

    Sur Linux et macOS, il existe une astuce utile pour simplifier cela.
Voyons comment s'y prendre, avec par exemple le programme
\texttt{fibonacci.py} que vous pouvez
\href{data/fibonacci.py}{télécharger ici} (nous avons vu ce code en
détail dans les deux compléments précédents). Commencez par sauver ce
code sur votre ordinateur dans un fichier qui s'appelle, bien entendu,
\texttt{fibonacci.py}.\\

    On commence par éditer le tout début du fichier pour lui ajouter une
\textbf{première ligne}~:

    \begin{verbatim}
#!/usr/bin/env python3

## La suite de Fibonacci (Suite)
...etc...
\end{verbatim}

    Cette première ligne s'appelle un
\href{http://en.wikipedia.org/wiki/Shebang_\%28Unix\%29}{Shebang} dans
le jargon Unix. Unix stipule que le Shebang doit être en
\textbf{première position} dans le fichier.\\

    Ensuite on rajoute au fichier, depuis le terminal, le caractère
exécutable comme ceci~:

    \begin{verbatim}
$ pwd
/le/chemin/jusqu/a/
\end{verbatim}

    \begin{verbatim}
$ chmod +x fibonacci.py
\end{verbatim}

    À partir de là, vous pouvez utiliser le fichier \texttt{fibonacci.py}
comme une commande, sans avoir à mentionner \texttt{python3}, qui sera
invoqué au travers du shebang~:

    \begin{verbatim}
$ /le/chemin/jusqu/a/fibonacci.py 20
fibonacci(20) = 10946
\end{verbatim}

    Et donc vous pouvez aussi le déplacer dans un répertoire qui est dans
votre variable \texttt{PATH}; de cette façon vous les rendez ainsi
accessible à partir n'importe quel répertoire en faisant simplement~:

    \begin{verbatim}
$ export PATH=/le/chemin/jusqu/a:$PATH
\end{verbatim}

    \begin{verbatim}
$ cd /tmp
$ fibonacci.py 20
fibonacci(20) = 10946
\end{verbatim}

    \textbf{Remarque~:} tout ceci fonctionne très bien tant que votre point
d'entrée - ici \texttt{fibonacci.py} - n'utilise que des modules
standards. Dans le cas où le point d'entrée vient avec au moins un
module, il est également nécessaire d'installer ces modules quelque
part, et d'indiquer au point d'entrée comment les trouver, nous y
reviendrons dans la semaine où nous parlerons des modules.