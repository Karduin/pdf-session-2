    \hypertarget{comptage-dans-les-chaines}{%
\section{Comptage dans les chaines}\label{comptage-dans-les-chaines}}

    \hypertarget{exercice---niveau-basique}{%
\subsection{Exercice - niveau basique}\label{exercice---niveau-basique}}

    Nous remercions Benoit Izac pour cette contribution aux exercices.

    \hypertarget{la-commande-unix-wc1}{%
\subsection{La commande UNIX wc(1)}\label{la-commande-unix-wc1}}

\begin{center}\rule{0.5\linewidth}{\linethickness}\end{center}

Sur les systèmes de type UNIX, la commande
\href{http://pubs.opengroup.org/onlinepubs/9699919799/utilities/wc.html}{wc}
permet de compter le nombre de lignes, de mots et d'octets (ou de
caractères) présents sur l'entrée standard ou contenus dans un fichier.\\

L'exercice consiste à écrire une fonction nommée \emph{wc} qui prendra
en argument une chaîne de caractères et retournera une liste contenant
trois éléments~:

\begin{enumerate}
\def\labelenumi{\arabic{enumi}.}
\tightlist
\item
  le nombre de lignes (plus précisément le nombre de retours à la
  ligne)~;
\item
  le nombre de mots (un mot étant séparé par des espaces)~;
\item
  le nombre de caractères (on utilisera uniquement le jeu de caractères
  ASCII).
\end{enumerate}

    \begin{Verbatim}[commandchars=\\\{\}]
{\color{incolor}In [{\color{incolor} }]:} \PY{c+c1}{\PYZsh{} chargement de l\PYZsq{}exercice}
        \PY{k+kn}{from} \PY{n+nn}{corrections}\PY{n+nn}{.}\PY{n+nn}{exo\PYZus{}wc} \PY{k}{import} \PY{n}{exo\PYZus{}wc}
\end{Verbatim}


    \begin{Verbatim}[commandchars=\\\{\}]
{\color{incolor}In [{\color{incolor} }]:} \PY{c+c1}{\PYZsh{} exemple}
        \PY{n}{exo\PYZus{}wc}\PY{o}{.}\PY{n}{example}\PY{p}{(}\PY{p}{)}
\end{Verbatim}


    \textbf{Indice}~: nous avons vu rapidement la boucle \texttt{for},
sachez toutefois qu'on peut tout à fait résoudre l'exercice en utilisant
uniquement la bibliothèque standard.\\

\textbf{Remarque}~: usuellement, ce genre de fonctions retournerait
plutôt un tuple qu'une liste, mais comme nous ne voyons les tuples que
la semaine prochaine..

    À vous de jouer~:

    \begin{Verbatim}[commandchars=\\\{\}]
{\color{incolor}In [{\color{incolor} }]:} \PY{c+c1}{\PYZsh{} la fonction à implémenter}
        \PY{k}{def} \PY{n+nf}{wc}\PY{p}{(}\PY{n}{string}\PY{p}{)}\PY{p}{:}
            \PY{c+c1}{\PYZsh{} remplacer pass par votre code}
            \PY{k}{pass}
\end{Verbatim}


    \begin{Verbatim}[commandchars=\\\{\}]
{\color{incolor}In [{\color{incolor} }]:} \PY{c+c1}{\PYZsh{} correction}
        \PY{n}{exo\PYZus{}wc}\PY{o}{.}\PY{n}{correction}\PY{p}{(}\PY{n}{wc}\PY{p}{)}
\end{Verbatim}