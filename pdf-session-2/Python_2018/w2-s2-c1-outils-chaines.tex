    \hypertarget{les-outils-de-base-sur-les-chauxeenes-de-caractuxe8res-str}{%
\section{\texorpdfstring{Les outils de base sur les chaînes de
caractères
(\texttt{str})}{Les outils de base sur les chaînes de caractères (str)}}\label{les-outils-de-base-sur-les-chauxeenes-de-caractuxe8res-str}}

    \hypertarget{compluxe9ment---niveau-intermuxe9diaire}{%
\subsection{Complément - niveau
intermédiaire}\label{compluxe9ment---niveau-intermuxe9diaire}}

    \hypertarget{lire-la-documentation}{%
\subsubsection{Lire la documentation}\label{lire-la-documentation}}

    Même après des années de pratique, il est difficile de se souvenir de
toutes les méthodes travaillant sur les chaînes de caractères. Aussi il
est toujours utile de recourir à la documentation embarquée

    \begin{Verbatim}[commandchars=\\\{\}]
{\color{incolor}In [{\color{incolor}1}]:} \PY{n}{help}\PY{p}{(}\PY{n+nb}{str}\PY{p}{)}
\end{Verbatim}


    \begin{Verbatim}[commandchars=\\\{\}]
Help on class str in module builtins:

class str(object)
 |  str(object='') -> str
 |  str(bytes\_or\_buffer[, encoding[, errors]]) -> str
 |  
 |  Create a new string object from the given object. If encoding or
 |  errors is specified, then the object must expose a data buffer
 |  that will be decoded using the given encoding and error handler.
 |  Otherwise, returns the result of object.\_\_str\_\_() (if defined)
 |  or repr(object).
 |  encoding defaults to sys.getdefaultencoding().
 |  errors defaults to 'strict'.
 |  
 |  Methods defined here:
 |  
 |  \_\_add\_\_(self, value, /)
 |      Return self+value.
 |  
 |  \_\_contains\_\_(self, key, /)
 |      Return key in self.
 |  
 |  \_\_eq\_\_(self, value, /)
 |      Return self==value.
 |  
 |  \_\_format\_\_({\ldots})
 |      S.\_\_format\_\_(format\_spec) -> str
 |      
 |      Return a formatted version of S as described by format\_spec.
 |  
 |  \_\_ge\_\_(self, value, /)
 |      Return self>=value.
 |  
 |  \_\_getattribute\_\_(self, name, /)
 |      Return getattr(self, name).
 |  
 |  \_\_getitem\_\_(self, key, /)
 |      Return self[key].
 |  
 |  \_\_getnewargs\_\_({\ldots})
 |  
 |  \_\_gt\_\_(self, value, /)
 |      Return self>value.
 |  
 |  \_\_hash\_\_(self, /)
 |      Return hash(self).
 |  
 |  \_\_iter\_\_(self, /)
 |      Implement iter(self).
 |  
 |  \_\_le\_\_(self, value, /)
 |      Return self<=value.
 |  
 |  \_\_len\_\_(self, /)
 |      Return len(self).
 |  
 |  \_\_lt\_\_(self, value, /)
 |      Return self<value.
 |  
 |  \_\_mod\_\_(self, value, /)
 |      Return self\%value.
 |  
 |  \_\_mul\_\_(self, value, /)
 |      Return self*value.n
 |  
 |  \_\_ne\_\_(self, value, /)
 |      Return self!=value.
 |  
 |  \_\_new\_\_(*args, **kwargs) from builtins.type
 |      Create and return a new object.  See help(type) for accurate signature.
 |  
 |  \_\_repr\_\_(self, /)
 |      Return repr(self).
 |  
 |  \_\_rmod\_\_(self, value, /)
 |      Return value\%self.
 |  
 |  \_\_rmul\_\_(self, value, /)
 |      Return self*value.
 |  
 |  \_\_sizeof\_\_({\ldots})
 |      S.\_\_sizeof\_\_() -> size of S in memory, in bytes
 |  
 |  \_\_str\_\_(self, /)
 |      Return str(self).
 |  
 |  capitalize({\ldots})
 |      S.capitalize() -> str
 |      
 |      Return a capitalized version of S, i.e. make the first character
 |      have upper case and the rest lower case.
 |  
 |  casefold({\ldots})
 |      S.casefold() -> str
 |      
 |      Return a version of S suitable for caseless comparisons.
 |  
 |  center({\ldots})
 |      S.center(width[, fillchar]) -> str
 |      
 |      Return S centered in a string of length width. Padding is
 |      done using the specified fill character (default is a space)
 |  
 |  count({\ldots})
 |      S.count(sub[, start[, end]]) -> int
 |      
 |      Return the number of non-overlapping occurrences of substring sub in
 |      string S[start:end].  Optional arguments start and end are
 |      interpreted as in slice notation.
 |  
 |  encode({\ldots})
 |      S.encode(encoding='utf-8', errors='strict') -> bytes
 |      
 |      Encode S using the codec registered for encoding. Default encoding
 |      is 'utf-8'. errors may be given to set a different error
 |      handling scheme. Default is 'strict' meaning that encoding errors raise
 |      a UnicodeEncodeError. Other possible values are 'ignore', 'replace' and
 |      'xmlcharrefreplace' as well as any other name registered with
 |      codecs.register\_error that can handle UnicodeEncodeErrors.
 |  
 |  endswith({\ldots})
 |      S.endswith(suffix[, start[, end]]) -> bool
 |      
 |      Return True if S ends with the specified suffix, False otherwise.
 |      With optional start, test S beginning at that position.
 |      With optional end, stop comparing S at that position.
 |      suffix can also be a tuple of strings to try.
 |  
 |  expandtabs({\ldots})
 |      S.expandtabs(tabsize=8) -> str
 |      
 |      Return a copy of S where all tab characters are expanded using spaces.
 |      If tabsize is not given, a tab size of 8 characters is assumed.
 |  
 |  find({\ldots})
 |      S.find(sub[, start[, end]]) -> int
 |      
 |      Return the lowest index in S where substring sub is found,
 |      such that sub is contained within S[start:end].  Optional
 |      arguments start and end are interpreted as in slice notation.
 |      
 |      Return -1 on failure.
 |  
 |  format({\ldots})
 |      S.format(*args, **kwargs) -> str
 |      
 |      Return a formatted version of S, using substitutions from args and kwargs.
 |      The substitutions are identified by braces ('\{' and '\}').
 |  
 |  format\_map({\ldots})
 |      S.format\_map(mapping) -> str
 |      
 |      Return a formatted version of S, using substitutions from mapping.
 |      The substitutions are identified by braces ('\{' and '\}').
 |  
 |  index({\ldots})
 |      S.index(sub[, start[, end]]) -> int
 |      
 |      Return the lowest index in S where substring sub is found, 
 |      such that sub is contained within S[start:end].  Optional
 |      arguments start and end are interpreted as in slice notation.
 |      
 |      Raises ValueError when the substring is not found.
 |  
 |  isalnum({\ldots})
 |      S.isalnum() -> bool
 |      
 |      Return True if all characters in S are alphanumeric
 |      and there is at least one character in S, False otherwise.
 |  
 |  isalpha({\ldots})
 |      S.isalpha() -> bool
 |      
 |      Return True if all characters in S are alphabetic
 |      and there is at least one character in S, False otherwise.
 |  
 |  isdecimal({\ldots})
 |      S.isdecimal() -> bool
 |      
 |      Return True if there are only decimal characters in S,
 |      False otherwise.
 |  
 |  isdigit({\ldots})
 |      S.isdigit() -> bool
 |      
 |      Return True if all characters in S are digits
 |      and there is at least one character in S, False otherwise.
 |  
 |  isidentifier({\ldots})
 |      S.isidentifier() -> bool
 |      
 |      Return True if S is a valid identifier according
 |      to the language definition.
 |      
 |      Use keyword.iskeyword() to test for reserved identifiers
 |      such as "def" and "class".
 |  
 |  islower({\ldots})
 |      S.islower() -> bool
 |      
 |      Return True if all cased characters in S are lowercase and there is
 |      at least one cased character in S, False otherwise.
 |  
 |  isnumeric({\ldots})
 |      S.isnumeric() -> bool
 |      
 |      Return True if there are only numeric characters in S,
 |      False otherwise.
 |  
 |  isprintable({\ldots})
 |      S.isprintable() -> bool
 |      
 |      Return True if all characters in S are considered
 |      printable in repr() or S is empty, False otherwise.
 |  
 |  isspace({\ldots})
 |      S.isspace() -> bool
 |      
 |      Return True if all characters in S are whitespace
 |      and there is at least one character in S, False otherwise.
 |  
 |  istitle({\ldots})
 |      S.istitle() -> bool
 |      
 |      Return True if S is a titlecased string and there is at least one
 |      character in S, i.e. upper- and titlecase characters may only
 |      follow uncased characters and lowercase characters only cased ones.
 |      Return False otherwise.
 |  
 |  isupper({\ldots})
 |      S.isupper() -> bool
 |      
 |      Return True if all cased characters in S are uppercase and there is
 |      at least one cased character in S, False otherwise.
 |  
 |  join({\ldots})
 |      S.join(iterable) -> str
 |      
 |      Return a string which is the concatenation of the strings in the
 |      iterable.  The separator between elements is S.
 |  
 |  ljust({\ldots})
 |      S.ljust(width[, fillchar]) -> str
 |      
 |      Return S left-justified in a Unicode string of length width. Padding is
 |      done using the specified fill character (default is a space).
 |  
 |  lower({\ldots})
 |      S.lower() -> str
 |      
 |      Return a copy of the string S converted to lowercase.
 |  
 |  lstrip({\ldots})
 |      S.lstrip([chars]) -> str
 |      
 |      Return a copy of the string S with leading whitespace removed.
 |      If chars is given and not None, remove characters in chars instead.
 |  
 |  partition({\ldots})
 |      S.partition(sep) -> (head, sep, tail)
 |      
 |      Search for the separator sep in S, and return the part before it,
 |      the separator itself, and the part after it.  If the separator is not
 |      found, return S and two empty strings.
 |  
 |  replace({\ldots})
 |      S.replace(old, new[, count]) -> str
 |      
 |      Return a copy of S with all occurrences of substring
 |      old replaced by new.  If the optional argument count is
 |      given, only the first count occurrences are replaced.
 |  
 |  rfind({\ldots})
 |      S.rfind(sub[, start[, end]]) -> int
 |      
 |      Return the highest index in S where substring sub is found,
 |      such that sub is contained within S[start:end].  Optional
 |      arguments start and end are interpreted as in slice notation.
 |      
 |      Return -1 on failure.
 |  
 |  rindex({\ldots})
 |      S.rindex(sub[, start[, end]]) -> int
 |      
 |      Return the highest index in S where substring sub is found,
 |      such that sub is contained within S[start:end].  Optional
 |      arguments start and end are interpreted as in slice notation.
 |      
 |      Raises ValueError when the substring is not found.
 |  
 |  rjust({\ldots})
 |      S.rjust(width[, fillchar]) -> str
 |      
 |      Return S right-justified in a string of length width. Padding is
 |      done using the specified fill character (default is a space).
 |  
 |  rpartition({\ldots})
 |      S.rpartition(sep) -> (head, sep, tail)
 |      
 |      Search for the separator sep in S, starting at the end of S, and return
 |      the part before it, the separator itself, and the part after it.  If the
 |      separator is not found, return two empty strings and S.
 |  
 |  rsplit({\ldots})
 |      S.rsplit(sep=None, maxsplit=-1) -> list of strings
 |      
 |      Return a list of the words in S, using sep as the
 |      delimiter string, starting at the end of the string and
 |      working to the front.  If maxsplit is given, at most maxsplit
 |      splits are done. If sep is not specified, any whitespace string
 |      is a separator.
 |  
 |  rstrip({\ldots})
 |      S.rstrip([chars]) -> str
 |      
 |      Return a copy of the string S with trailing whitespace removed.
 |      If chars is given and not None, remove characters in chars instead.
 |  
 |  split({\ldots})
 |      S.split(sep=None, maxsplit=-1) -> list of strings
 |      
 |      Return a list of the words in S, using sep as the
 |      delimiter string.  If maxsplit is given, at most maxsplit
 |      splits are done. If sep is not specified or is None, any
 |      whitespace string is a separator and empty strings are
 |      removed from the result.
 |  
 |  splitlines({\ldots})
 |      S.splitlines([keepends]) -> list of strings
 |      
 |      Return a list of the lines in S, breaking at line boundaries.
 |      Line breaks are not included in the resulting list unless keepends
 |      is given and true.
 |  
 |  startswith({\ldots})
 |      S.startswith(prefix[, start[, end]]) -> bool
 |      
 |      Return True if S starts with the specified prefix, False otherwise.
 |      With optional start, test S beginning at that position.
 |      With optional end, stop comparing S at that position.
 |      prefix can also be a tuple of strings to try.
 |  
 |  strip({\ldots})
 |      S.strip([chars]) -> str
 |      
 |      Return a copy of the string S with leading and trailing
 |      whitespace removed.
 |      If chars is given and not None, remove characters in chars instead.
 |  
 |  swapcase({\ldots})
 |      S.swapcase() -> str
 |      
 |      Return a copy of S with uppercase characters converted to lowercase
 |      and vice versa.
 |  
 |  title({\ldots})
 |      S.title() -> str
 |      
 |      Return a titlecased version of S, i.e. words start with title case
 |      characters, all remaining cased characters have lower case.
 |  
 |  translate({\ldots})
 |      S.translate(table) -> str
 |      
 |      Return a copy of the string S in which each character has been mapped
 |      through the given translation table. The table must implement
 |      lookup/indexing via \_\_getitem\_\_, for instance a dictionary or list,
 |      mapping Unicode ordinals to Unicode ordinals, strings, or None. If
 |      this operation raises LookupError, the character is left untouched.
 |      Characters mapped to None are deleted.
 |  
 |  upper({\ldots})
 |      S.upper() -> str
 |      
 |      Return a copy of S converted to uppercase.
 |  
 |  zfill({\ldots})
 |      S.zfill(width) -> str
 |      
 |      Pad a numeric string S with zeros on the left, to fill a field
 |      of the specified width. The string S is never truncated.
 |  
 |  ----------------------------------------------------------------------
 |  Static methods defined here:
 |  
 |  maketrans(x, y=None, z=None, /)
 |      Return a translation table usable for str.translate().
 |      
 |      If there is only one argument, it must be a dictionary mapping Unicode
 |      ordinals (integers) or characters to Unicode ordinals, strings or None.
 |      Character keys will be then converted to ordinals.
 |      If there are two arguments, they must be strings of equal length, and
 |      in the resulting dictionary, each character in x will be mapped to the
 |      character at the same position in y. If there is a third argument, it
 |      must be a string, whose characters will be mapped to None in the result.


    \end{Verbatim}
\newpage
    Nous allons tenter ici de citer les méthodes les plus utilisées. Nous
n'avons le temps que de les utiliser de manière très simple, mais bien
souvent il est possible de passer en argument des options permettant de
ne travailler que sur une sous-chaîne, ou sur la première ou dernière
occurrence d'une sous-chaîne. Nous vous renvoyons à la documentation
pour obtenir toutes les précisions utiles.

    \hypertarget{duxe9coupage---assemblage-split-et-join}{%
\subsubsection{\texorpdfstring{Découpage - assemblage~: \texttt{split}
et
\texttt{join}}{Découpage - assemblage~: split et join}}\label{duxe9coupage---assemblage-split-et-join}}

    Les méthodes \texttt{split} et \texttt{join} permettent de découper une
chaîne selon un séparateur pour obtenir une liste, et à l'inverse de
reconstruire une chaîne à partir d'une liste.

    \texttt{split} permet donc de découper~:

    \begin{Verbatim}[commandchars=\\\{\}]
{\color{incolor}In [{\color{incolor}2}]:} \PY{l+s+s1}{\PYZsq{}}\PY{l+s+s1}{abc=:=def=:=ghi=:=jkl}\PY{l+s+s1}{\PYZsq{}}\PY{o}{.}\PY{n}{split}\PY{p}{(}\PY{l+s+s1}{\PYZsq{}}\PY{l+s+s1}{=:=}\PY{l+s+s1}{\PYZsq{}}\PY{p}{)}
\end{Verbatim}


\begin{Verbatim}[commandchars=\\\{\}]
{\color{outcolor}Out[{\color{outcolor}2}]:} ['abc', 'def', 'ghi', 'jkl']
\end{Verbatim}
            
    Et à l'inverse~:

    \begin{Verbatim}[commandchars=\\\{\}]
{\color{incolor}In [{\color{incolor}3}]:} \PY{l+s+s2}{\PYZdq{}}\PY{l+s+s2}{=:=}\PY{l+s+s2}{\PYZdq{}}\PY{o}{.}\PY{n}{join}\PY{p}{(}\PY{p}{[}\PY{l+s+s1}{\PYZsq{}}\PY{l+s+s1}{abc}\PY{l+s+s1}{\PYZsq{}}\PY{p}{,} \PY{l+s+s1}{\PYZsq{}}\PY{l+s+s1}{def}\PY{l+s+s1}{\PYZsq{}}\PY{p}{,} \PY{l+s+s1}{\PYZsq{}}\PY{l+s+s1}{ghi}\PY{l+s+s1}{\PYZsq{}}\PY{p}{,} \PY{l+s+s1}{\PYZsq{}}\PY{l+s+s1}{jkl}\PY{l+s+s1}{\PYZsq{}}\PY{p}{]}\PY{p}{)}
\end{Verbatim}


\begin{Verbatim}[commandchars=\\\{\}]
{\color{outcolor}Out[{\color{outcolor}3}]:} 'abc=:=def=:=ghi=:=jkl'
\end{Verbatim}
            
    Attention toutefois si le séparateur est un terminateur, la liste
résultat contient alors une dernière chaîne vide. En pratique, on
utilisera la méthode \texttt{strip}, que nous allons voir ci-dessous,
avant la méthode \texttt{split} pour éviter ce problème.

    \begin{Verbatim}[commandchars=\\\{\}]
{\color{incolor}In [{\color{incolor}4}]:} \PY{l+s+s1}{\PYZsq{}}\PY{l+s+s1}{abc;def;ghi;jkl;}\PY{l+s+s1}{\PYZsq{}}\PY{o}{.}\PY{n}{split}\PY{p}{(}\PY{l+s+s1}{\PYZsq{}}\PY{l+s+s1}{;}\PY{l+s+s1}{\PYZsq{}}\PY{p}{)}
\end{Verbatim}


\begin{Verbatim}[commandchars=\\\{\}]
{\color{outcolor}Out[{\color{outcolor}4}]:} ['abc', 'def', 'ghi', 'jkl', '']
\end{Verbatim}
            
    Qui s'inverse correctement cependant~:

    \begin{Verbatim}[commandchars=\\\{\}]
{\color{incolor}In [{\color{incolor}5}]:} \PY{l+s+s2}{\PYZdq{}}\PY{l+s+s2}{;}\PY{l+s+s2}{\PYZdq{}}\PY{o}{.}\PY{n}{join}\PY{p}{(}\PY{p}{[}\PY{l+s+s1}{\PYZsq{}}\PY{l+s+s1}{abc}\PY{l+s+s1}{\PYZsq{}}\PY{p}{,} \PY{l+s+s1}{\PYZsq{}}\PY{l+s+s1}{def}\PY{l+s+s1}{\PYZsq{}}\PY{p}{,} \PY{l+s+s1}{\PYZsq{}}\PY{l+s+s1}{ghi}\PY{l+s+s1}{\PYZsq{}}\PY{p}{,} \PY{l+s+s1}{\PYZsq{}}\PY{l+s+s1}{jkl}\PY{l+s+s1}{\PYZsq{}}\PY{p}{,} \PY{l+s+s1}{\PYZsq{}}\PY{l+s+s1}{\PYZsq{}}\PY{p}{]}\PY{p}{)}
\end{Verbatim}


\begin{Verbatim}[commandchars=\\\{\}]
{\color{outcolor}Out[{\color{outcolor}5}]:} 'abc;def;ghi;jkl;'
\end{Verbatim}
            
    \hypertarget{remplacement-replace}{%
\subsubsection{\texorpdfstring{Remplacement~:
\texttt{replace}}{Remplacement~: replace}}\label{remplacement-replace}}

    \texttt{replace} est très pratique pour remplacer une sous-chaîne par
une autre, avec une limite éventuelle sur le nombre de remplacements~:

    \begin{Verbatim}[commandchars=\\\{\}]
{\color{incolor}In [{\color{incolor}6}]:} \PY{l+s+s2}{\PYZdq{}}\PY{l+s+s2}{abcdefabcdefabcdef}\PY{l+s+s2}{\PYZdq{}}\PY{o}{.}\PY{n}{replace}\PY{p}{(}\PY{l+s+s2}{\PYZdq{}}\PY{l+s+s2}{abc}\PY{l+s+s2}{\PYZdq{}}\PY{p}{,} \PY{l+s+s2}{\PYZdq{}}\PY{l+s+s2}{zoo}\PY{l+s+s2}{\PYZdq{}}\PY{p}{)}
\end{Verbatim}


\begin{Verbatim}[commandchars=\\\{\}]
{\color{outcolor}Out[{\color{outcolor}6}]:} 'zoodefzoodefzoodef'
\end{Verbatim}
            
    \begin{Verbatim}[commandchars=\\\{\}]
{\color{incolor}In [{\color{incolor}7}]:} \PY{l+s+s2}{\PYZdq{}}\PY{l+s+s2}{abcdefabcdefabcdef}\PY{l+s+s2}{\PYZdq{}}\PY{o}{.}\PY{n}{replace}\PY{p}{(}\PY{l+s+s2}{\PYZdq{}}\PY{l+s+s2}{abc}\PY{l+s+s2}{\PYZdq{}}\PY{p}{,} \PY{l+s+s2}{\PYZdq{}}\PY{l+s+s2}{zoo}\PY{l+s+s2}{\PYZdq{}}\PY{p}{,} \PY{l+m+mi}{2}\PY{p}{)}
\end{Verbatim}


\begin{Verbatim}[commandchars=\\\{\}]
{\color{outcolor}Out[{\color{outcolor}7}]:} 'zoodefzoodefabcdef'
\end{Verbatim}
            
    Plusieurs appels à \texttt{replace} peuvent être chaînés comme ceci~:

    \begin{Verbatim}[commandchars=\\\{\}]
{\color{incolor}In [{\color{incolor}8}]:} \PY{l+s+s2}{\PYZdq{}}\PY{l+s+s2}{les [x] qui disent [y]}\PY{l+s+s2}{\PYZdq{}}\PY{o}{.}\PY{n}{replace}\PY{p}{(}\PY{l+s+s2}{\PYZdq{}}\PY{l+s+s2}{[x]}\PY{l+s+s2}{\PYZdq{}}\PY{p}{,} \PY{l+s+s2}{\PYZdq{}}\PY{l+s+s2}{chevaliers}\PY{l+s+s2}{\PYZdq{}}\PY{p}{)}\PY{o}{.}\PY{n}{replace}\PY{p}{(}\PY{l+s+s2}{\PYZdq{}}\PY{l+s+s2}{[y]}\PY{l+s+s2}{\PYZdq{}}\PY{p}{,} \PY{l+s+s2}{\PYZdq{}}\PY{l+s+s2}{Ni}\PY{l+s+s2}{\PYZdq{}}\PY{p}{)}
\end{Verbatim}


\begin{Verbatim}[commandchars=\\\{\}]
{\color{outcolor}Out[{\color{outcolor}8}]:} 'les chevaliers qui disent Ni'
\end{Verbatim}
            
    \hypertarget{nettoyage-strip}{%
\subsubsection{\texorpdfstring{Nettoyage~:
\texttt{strip}}{Nettoyage~: strip}}\label{nettoyage-strip}}

    On pourrait par exemple utiliser \texttt{replace} pour enlever les
espaces dans une chaîne, ce qui peut être utile pour ``nettoyer'' comme
ceci~:

    \begin{Verbatim}[commandchars=\\\{\}]
{\color{incolor}In [{\color{incolor}9}]:} \PY{l+s+s2}{\PYZdq{}}\PY{l+s+s2}{ abc:def:ghi }\PY{l+s+s2}{\PYZdq{}}\PY{o}{.}\PY{n}{replace}\PY{p}{(}\PY{l+s+s2}{\PYZdq{}}\PY{l+s+s2}{ }\PY{l+s+s2}{\PYZdq{}}\PY{p}{,} \PY{l+s+s2}{\PYZdq{}}\PY{l+s+s2}{\PYZdq{}}\PY{p}{)}
\end{Verbatim}


\begin{Verbatim}[commandchars=\\\{\}]
{\color{outcolor}Out[{\color{outcolor}9}]:} 'abc:def:ghi'
\end{Verbatim}
            
    Toutefois bien souvent on préfère utiliser \texttt{strip} qui ne
s'occupe que du début et de la fin de la chaîne, et gère aussi les
tabulations et autres retour à la ligne~:

    \begin{Verbatim}[commandchars=\\\{\}]
{\color{incolor}In [{\color{incolor}10}]:} \PY{l+s+s2}{\PYZdq{}}\PY{l+s+s2}{ }\PY{l+s+se}{\PYZbs{}t}\PY{l+s+s2}{une chaîne avec des trucs qui dépassent }\PY{l+s+se}{\PYZbs{}n}\PY{l+s+s2}{\PYZdq{}}\PY{o}{.}\PY{n}{strip}\PY{p}{(}\PY{p}{)}
\end{Verbatim}


\begin{Verbatim}[commandchars=\\\{\}]
{\color{outcolor}Out[{\color{outcolor}10}]:} 'une chaîne avec des trucs qui dépassent'
\end{Verbatim}
            
    On peut appliquer \texttt{strip} avant \texttt{split} pour éviter le
problème du dernier élément vide~:

    \begin{Verbatim}[commandchars=\\\{\}]
{\color{incolor}In [{\color{incolor}11}]:} \PY{l+s+s1}{\PYZsq{}}\PY{l+s+s1}{abc;def;ghi;jkl;}\PY{l+s+s1}{\PYZsq{}}\PY{o}{.}\PY{n}{strip}\PY{p}{(}\PY{l+s+s1}{\PYZsq{}}\PY{l+s+s1}{;}\PY{l+s+s1}{\PYZsq{}}\PY{p}{)}\PY{o}{.}\PY{n}{split}\PY{p}{(}\PY{l+s+s1}{\PYZsq{}}\PY{l+s+s1}{;}\PY{l+s+s1}{\PYZsq{}}\PY{p}{)}
\end{Verbatim}


\begin{Verbatim}[commandchars=\\\{\}]
{\color{outcolor}Out[{\color{outcolor}11}]:} ['abc', 'def', 'ghi', 'jkl']
\end{Verbatim}
            
    \hypertarget{rechercher-une-sous-chauxeene}{%
\subsubsection{Rechercher une
sous-chaîne}\label{rechercher-une-sous-chauxeene}}

    Plusieurs outils permettent de chercher une sous-chaîne. Il existe
\texttt{find} qui renvoie le plus petit index où on trouve la
sous-chaîne~:

    \begin{Verbatim}[commandchars=\\\{\}]
{\color{incolor}In [{\color{incolor}12}]:} \PY{c+c1}{\PYZsh{} l\PYZsq{}indice du début de la première occurrence}
         \PY{l+s+s2}{\PYZdq{}}\PY{l+s+s2}{abcdefcdefghefghijk}\PY{l+s+s2}{\PYZdq{}}\PY{o}{.}\PY{n}{find}\PY{p}{(}\PY{l+s+s2}{\PYZdq{}}\PY{l+s+s2}{def}\PY{l+s+s2}{\PYZdq{}}\PY{p}{)}
\end{Verbatim}


\begin{Verbatim}[commandchars=\\\{\}]
{\color{outcolor}Out[{\color{outcolor}12}]:} 3
\end{Verbatim}
            
    \begin{Verbatim}[commandchars=\\\{\}]
{\color{incolor}In [{\color{incolor}13}]:} \PY{c+c1}{\PYZsh{} ou \PYZhy{}1 si la chaîne n\PYZsq{}est pas présente}
         \PY{l+s+s2}{\PYZdq{}}\PY{l+s+s2}{abcdefcdefghefghijk}\PY{l+s+s2}{\PYZdq{}}\PY{o}{.}\PY{n}{find}\PY{p}{(}\PY{l+s+s2}{\PYZdq{}}\PY{l+s+s2}{zoo}\PY{l+s+s2}{\PYZdq{}}\PY{p}{)}
\end{Verbatim}


\begin{Verbatim}[commandchars=\\\{\}]
{\color{outcolor}Out[{\color{outcolor}13}]:} -1
\end{Verbatim}
            
    \texttt{rfind} fonctionne comme \texttt{find} mais en partant de la fin
de la chaîne~:

    \begin{Verbatim}[commandchars=\\\{\}]
{\color{incolor}In [{\color{incolor}14}]:} \PY{c+c1}{\PYZsh{} en partant de la fin}
         \PY{l+s+s2}{\PYZdq{}}\PY{l+s+s2}{abcdefcdefghefghijk}\PY{l+s+s2}{\PYZdq{}}\PY{o}{.}\PY{n}{rfind}\PY{p}{(}\PY{l+s+s2}{\PYZdq{}}\PY{l+s+s2}{fgh}\PY{l+s+s2}{\PYZdq{}}\PY{p}{)}
\end{Verbatim}


\begin{Verbatim}[commandchars=\\\{\}]
{\color{outcolor}Out[{\color{outcolor}14}]:} 13
\end{Verbatim}
            
    \begin{Verbatim}[commandchars=\\\{\}]
{\color{incolor}In [{\color{incolor}15}]:} \PY{c+c1}{\PYZsh{} notez que le résultat correspond}
         \PY{c+c1}{\PYZsh{} tout de même toujours au début de la chaîne}
         \PY{l+s+s2}{\PYZdq{}}\PY{l+s+s2}{abcdefcdefghefghijk}\PY{l+s+s2}{\PYZdq{}}\PY{p}{[}\PY{l+m+mi}{13}\PY{p}{]}
\end{Verbatim}


\begin{Verbatim}[commandchars=\\\{\}]
{\color{outcolor}Out[{\color{outcolor}15}]:} 'f'
\end{Verbatim}
            
    La méthode \texttt{index} se comporte comme \texttt{find}, mais en cas
d'absence elle lève une \textbf{exception} (nous verrons ce concept plus
tard) plutôt que de renvoyer \texttt{-1}~:

    \begin{Verbatim}[commandchars=\\\{\}]
{\color{incolor}In [{\color{incolor}16}]:} \PY{l+s+s2}{\PYZdq{}}\PY{l+s+s2}{abcdefcdefghefghijk}\PY{l+s+s2}{\PYZdq{}}\PY{o}{.}\PY{n}{index}\PY{p}{(}\PY{l+s+s2}{\PYZdq{}}\PY{l+s+s2}{def}\PY{l+s+s2}{\PYZdq{}}\PY{p}{)}
\end{Verbatim}


\begin{Verbatim}[commandchars=\\\{\}]
{\color{outcolor}Out[{\color{outcolor}16}]:} 3
\end{Verbatim}
            
    \begin{Verbatim}[commandchars=\\\{\}]
{\color{incolor}In [{\color{incolor}17}]:} \PY{k}{try}\PY{p}{:}
             \PY{l+s+s2}{\PYZdq{}}\PY{l+s+s2}{abcdefcdefghefghijk}\PY{l+s+s2}{\PYZdq{}}\PY{o}{.}\PY{n}{index}\PY{p}{(}\PY{l+s+s2}{\PYZdq{}}\PY{l+s+s2}{zoo}\PY{l+s+s2}{\PYZdq{}}\PY{p}{)}
         \PY{k}{except} \PY{n+ne}{Exception} \PY{k}{as} \PY{n}{e}\PY{p}{:}
             \PY{n+nb}{print}\PY{p}{(}\PY{l+s+s2}{\PYZdq{}}\PY{l+s+s2}{OOPS}\PY{l+s+s2}{\PYZdq{}}\PY{p}{,} \PY{n+nb}{type}\PY{p}{(}\PY{n}{e}\PY{p}{)}\PY{p}{,} \PY{n}{e}\PY{p}{)}
\end{Verbatim}


    \begin{Verbatim}[commandchars=\\\{\}]
OOPS <class 'ValueError'> substring not found

    \end{Verbatim}

    Mais le plus simple pour chercher si une sous-chaîne est dans une autre
chaîne est d'utiliser l'instruction \texttt{in} sur laquelle nous
reviendrons lorsque nous parlerons des séquences~:

    \begin{Verbatim}[commandchars=\\\{\}]
{\color{incolor}In [{\color{incolor}18}]:} \PY{l+s+s2}{\PYZdq{}}\PY{l+s+s2}{def}\PY{l+s+s2}{\PYZdq{}} \PY{o+ow}{in} \PY{l+s+s2}{\PYZdq{}}\PY{l+s+s2}{abcdefcdefghefghijk}\PY{l+s+s2}{\PYZdq{}}
\end{Verbatim}


\begin{Verbatim}[commandchars=\\\{\}]
{\color{outcolor}Out[{\color{outcolor}18}]:} True
\end{Verbatim}
            
    La méthode \texttt{count} compte le nombre d'occurrences d'une
sous-chaîne~:

    \begin{Verbatim}[commandchars=\\\{\}]
{\color{incolor}In [{\color{incolor}19}]:} \PY{l+s+s2}{\PYZdq{}}\PY{l+s+s2}{abcdefcdefghefghijk}\PY{l+s+s2}{\PYZdq{}}\PY{o}{.}\PY{n}{count}\PY{p}{(}\PY{l+s+s2}{\PYZdq{}}\PY{l+s+s2}{ef}\PY{l+s+s2}{\PYZdq{}}\PY{p}{)}
\end{Verbatim}


\begin{Verbatim}[commandchars=\\\{\}]
{\color{outcolor}Out[{\color{outcolor}19}]:} 3
\end{Verbatim}
            
    Signalons enfin les méthodes de commodité suivantes~:

    \begin{Verbatim}[commandchars=\\\{\}]
{\color{incolor}In [{\color{incolor}20}]:} \PY{l+s+s2}{\PYZdq{}}\PY{l+s+s2}{abcdefcdefghefghijk}\PY{l+s+s2}{\PYZdq{}}\PY{o}{.}\PY{n}{startswith}\PY{p}{(}\PY{l+s+s2}{\PYZdq{}}\PY{l+s+s2}{abcd}\PY{l+s+s2}{\PYZdq{}}\PY{p}{)}
\end{Verbatim}


\begin{Verbatim}[commandchars=\\\{\}]
{\color{outcolor}Out[{\color{outcolor}20}]:} True
\end{Verbatim}
            
    \begin{Verbatim}[commandchars=\\\{\}]
{\color{incolor}In [{\color{incolor}21}]:} \PY{l+s+s2}{\PYZdq{}}\PY{l+s+s2}{abcdefcdefghefghijk}\PY{l+s+s2}{\PYZdq{}}\PY{o}{.}\PY{n}{endswith}\PY{p}{(}\PY{l+s+s2}{\PYZdq{}}\PY{l+s+s2}{ghijk}\PY{l+s+s2}{\PYZdq{}}\PY{p}{)}
\end{Verbatim}


\begin{Verbatim}[commandchars=\\\{\}]
{\color{outcolor}Out[{\color{outcolor}21}]:} True
\end{Verbatim}
            
    S'agissant des deux dernières, remarquons que~:

    \texttt{chaine.startswith(sous\_chaine)} \(\Longleftrightarrow\)
\texttt{chaine.find(sous\_chaine)\ ==\ 0}

\texttt{chaine.endswith(sous\_chaine)} \(\Longleftrightarrow\)
\texttt{chaine.rfind(sous\_chaine)\ ==\ (len(chaine)\ -\ len(sous\_chaine))}

    On remarque ici la supériorité en terme d'expressivité des méthodes
pythoniques \texttt{startswith} et \texttt{endswith}.

    \hypertarget{changement-de-casse}{%
\subsubsection{Changement de casse}\label{changement-de-casse}}

    Voici pour conclure quelques méthodes utiles qui parlent d'elles-mêmes~:

    \begin{Verbatim}[commandchars=\\\{\}]
{\color{incolor}In [{\color{incolor}22}]:} \PY{l+s+s2}{\PYZdq{}}\PY{l+s+s2}{monty PYTHON}\PY{l+s+s2}{\PYZdq{}}\PY{o}{.}\PY{n}{upper}\PY{p}{(}\PY{p}{)}
\end{Verbatim}


\begin{Verbatim}[commandchars=\\\{\}]
{\color{outcolor}Out[{\color{outcolor}22}]:} 'MONTY PYTHON'
\end{Verbatim}
            
    \begin{Verbatim}[commandchars=\\\{\}]
{\color{incolor}In [{\color{incolor}23}]:} \PY{l+s+s2}{\PYZdq{}}\PY{l+s+s2}{monty PYTHON}\PY{l+s+s2}{\PYZdq{}}\PY{o}{.}\PY{n}{lower}\PY{p}{(}\PY{p}{)}
\end{Verbatim}


\begin{Verbatim}[commandchars=\\\{\}]
{\color{outcolor}Out[{\color{outcolor}23}]:} 'monty python'
\end{Verbatim}
            
    \begin{Verbatim}[commandchars=\\\{\}]
{\color{incolor}In [{\color{incolor}24}]:} \PY{l+s+s2}{\PYZdq{}}\PY{l+s+s2}{monty PYTHON}\PY{l+s+s2}{\PYZdq{}}\PY{o}{.}\PY{n}{swapcase}\PY{p}{(}\PY{p}{)}
\end{Verbatim}


\begin{Verbatim}[commandchars=\\\{\}]
{\color{outcolor}Out[{\color{outcolor}24}]:} 'MONTY python'
\end{Verbatim}
            
    \begin{Verbatim}[commandchars=\\\{\}]
{\color{incolor}In [{\color{incolor}25}]:} \PY{l+s+s2}{\PYZdq{}}\PY{l+s+s2}{monty PYTHON}\PY{l+s+s2}{\PYZdq{}}\PY{o}{.}\PY{n}{capitalize}\PY{p}{(}\PY{p}{)}
\end{Verbatim}


\begin{Verbatim}[commandchars=\\\{\}]
{\color{outcolor}Out[{\color{outcolor}25}]:} 'Monty python'
\end{Verbatim}
            
    \begin{Verbatim}[commandchars=\\\{\}]
{\color{incolor}In [{\color{incolor}26}]:} \PY{l+s+s2}{\PYZdq{}}\PY{l+s+s2}{monty PYTHON}\PY{l+s+s2}{\PYZdq{}}\PY{o}{.}\PY{n}{title}\PY{p}{(}\PY{p}{)}
\end{Verbatim}


\begin{Verbatim}[commandchars=\\\{\}]
{\color{outcolor}Out[{\color{outcolor}26}]:} 'Monty Python'
\end{Verbatim}
            
    \hypertarget{pour-en-savoir-plus}{%
\subsubsection{Pour en savoir plus}\label{pour-en-savoir-plus}}

    Tous ces outils sont
\href{https://docs.python.org/3/library/stdtypes.html\#string-methods}{documentés
en détail ici (en anglais)}.