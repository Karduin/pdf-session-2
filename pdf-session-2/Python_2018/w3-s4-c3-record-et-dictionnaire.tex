    \hypertarget{guxe9rer-des-enregistrements}{%
\section{Gérer des enregistrements}\label{guxe9rer-des-enregistrements}}

    \hypertarget{compluxe9ment---niveau-intermuxe9diaire}{%
\subsection{Complément - niveau
intermédiaire}\label{compluxe9ment---niveau-intermuxe9diaire}}

    \hypertarget{impluxe9menter-un-enregistrement-comme-un-dictionnaire}{%
\subsubsection{Implémenter un enregistrement comme un
dictionnaire}\label{impluxe9menter-un-enregistrement-comme-un-dictionnaire}}

    Il nous faut faire le lien entre dictionnaire Python et la notion
d'enregistrement, c'est-à-dire une donnée composite qui contient
plusieurs champs. (À cette notion correspond, selon les langages, ce
qu'on appelle un \texttt{struct} ou un \texttt{record}.)\\

Imaginons qu'on veuille manipuler un ensemble de données concernant des
personnes~; chaque personne est supposée avoir un nom, un âge et une
adresse mail.\\

Il est possible, et assez fréquent, d'utiliser le dictionnaire comme
support pour modéliser ces données comme ceci~:

    \begin{Verbatim}[commandchars=\\\{\}]
{\color{incolor}In [{\color{incolor}1}]:} \PY{n}{personnes} \PY{o}{=} \PY{p}{[}
            \PY{p}{\PYZob{}}\PY{l+s+s1}{\PYZsq{}}\PY{l+s+s1}{nom}\PY{l+s+s1}{\PYZsq{}}\PY{p}{:} \PY{l+s+s1}{\PYZsq{}}\PY{l+s+s1}{Pierre}\PY{l+s+s1}{\PYZsq{}}\PY{p}{,}  \PY{l+s+s1}{\PYZsq{}}\PY{l+s+s1}{age}\PY{l+s+s1}{\PYZsq{}}\PY{p}{:} \PY{l+m+mi}{25}\PY{p}{,} \PY{l+s+s1}{\PYZsq{}}\PY{l+s+s1}{email}\PY{l+s+s1}{\PYZsq{}}\PY{p}{:} \PY{l+s+s1}{\PYZsq{}}\PY{l+s+s1}{pierre@example.com}\PY{l+s+s1}{\PYZsq{}}\PY{p}{\PYZcb{}}\PY{p}{,}
            \PY{p}{\PYZob{}}\PY{l+s+s1}{\PYZsq{}}\PY{l+s+s1}{nom}\PY{l+s+s1}{\PYZsq{}}\PY{p}{:} \PY{l+s+s1}{\PYZsq{}}\PY{l+s+s1}{Paul}\PY{l+s+s1}{\PYZsq{}}\PY{p}{,}    \PY{l+s+s1}{\PYZsq{}}\PY{l+s+s1}{age}\PY{l+s+s1}{\PYZsq{}}\PY{p}{:} \PY{l+m+mi}{18}\PY{p}{,} \PY{l+s+s1}{\PYZsq{}}\PY{l+s+s1}{email}\PY{l+s+s1}{\PYZsq{}}\PY{p}{:} \PY{l+s+s1}{\PYZsq{}}\PY{l+s+s1}{paul@example.com}\PY{l+s+s1}{\PYZsq{}}\PY{p}{\PYZcb{}}\PY{p}{,}
            \PY{p}{\PYZob{}}\PY{l+s+s1}{\PYZsq{}}\PY{l+s+s1}{nom}\PY{l+s+s1}{\PYZsq{}}\PY{p}{:} \PY{l+s+s1}{\PYZsq{}}\PY{l+s+s1}{Jacques}\PY{l+s+s1}{\PYZsq{}}\PY{p}{,} \PY{l+s+s1}{\PYZsq{}}\PY{l+s+s1}{age}\PY{l+s+s1}{\PYZsq{}}\PY{p}{:} \PY{l+m+mi}{52}\PY{p}{,} \PY{l+s+s1}{\PYZsq{}}\PY{l+s+s1}{email}\PY{l+s+s1}{\PYZsq{}}\PY{p}{:} \PY{l+s+s1}{\PYZsq{}}\PY{l+s+s1}{jacques@example.com}\PY{l+s+s1}{\PYZsq{}}\PY{p}{\PYZcb{}}\PY{p}{,}
        \PY{p}{]}
\end{Verbatim}


    Bon, très bien, nous avons nos données, il est facile de les utiliser.\\

Par exemple, pour l'anniversaire de Pierre on fera~:

    \begin{Verbatim}[commandchars=\\\{\}]
{\color{incolor}In [{\color{incolor}2}]:} \PY{n}{personnes}\PY{p}{[}\PY{l+m+mi}{0}\PY{p}{]}\PY{p}{[}\PY{l+s+s1}{\PYZsq{}}\PY{l+s+s1}{age}\PY{l+s+s1}{\PYZsq{}}\PY{p}{]} \PY{o}{+}\PY{o}{=} \PY{l+m+mi}{1}
\end{Verbatim}


    Ce qui nous donne~:

    \begin{Verbatim}[commandchars=\\\{\}]
{\color{incolor}In [{\color{incolor}3}]:} \PY{k}{for} \PY{n}{personne} \PY{o+ow}{in} \PY{n}{personnes}\PY{p}{:}
            \PY{n+nb}{print}\PY{p}{(}\PY{l+m+mi}{10}\PY{o}{*}\PY{l+s+s2}{\PYZdq{}}\PY{l+s+s2}{=}\PY{l+s+s2}{\PYZdq{}}\PY{p}{)}
            \PY{k}{for} \PY{n}{info}\PY{p}{,} \PY{n}{valeur} \PY{o+ow}{in} \PY{n}{personne}\PY{o}{.}\PY{n}{items}\PY{p}{(}\PY{p}{)}\PY{p}{:}
                \PY{n+nb}{print}\PY{p}{(}\PY{n}{f}\PY{l+s+s2}{\PYZdq{}}\PY{l+s+si}{\PYZob{}info\PYZcb{}}\PY{l+s+s2}{ \PYZhy{}\PYZgt{} }\PY{l+s+si}{\PYZob{}valeur\PYZcb{}}\PY{l+s+s2}{\PYZdq{}}\PY{p}{)}
\end{Verbatim}


    \begin{Verbatim}[commandchars=\\\{\}]
==========
nom -> Pierre
age -> 26
email -> pierre@example.com
==========
nom -> Paul
age -> 18
email -> paul@example.com
==========
nom -> Jacques
age -> 52
email -> jacques@example.com

    \end{Verbatim}

    \hypertarget{un-dictionnaire-pour-indexer-les-enregistrements}{%
\subsubsection{Un dictionnaire pour indexer les
enregistrements}\label{un-dictionnaire-pour-indexer-les-enregistrements}}

    Cela dit, il est bien clair que cette façon de faire n'est pas très
pratique~; pour marquer l'anniversaire de Pierre on ne sait bien entendu
pas que son enregistrement est le premier dans la liste. C'est pourquoi
il est plus adapté, pour modéliser ces informations, d'utiliser non pas
une liste, mais à nouveau\ldots{} un dictionnaire.\\

Si on imagine qu'on a commencé par lire ces données séquentiellement
dans un fichier, et qu'on a calculé l'objet \texttt{personnes} comme la
liste qu'on a vue ci-dessus, alors il est possible de construire un
index de ces dictionnaires, (un dictionnaire de dictionnaires, donc).\\

C'est-à-dire, en anticipant un peu sur la construction de dictionnaires
par compréhension~:

    \begin{Verbatim}[commandchars=\\\{\}]
{\color{incolor}In [{\color{incolor}4}]:} \PY{c+c1}{\PYZsh{} on crée un index permettant de retrouver rapidement}
        \PY{c+c1}{\PYZsh{} une personne dans la liste}
        \PY{n}{index\PYZus{}par\PYZus{}nom} \PY{o}{=} \PY{p}{\PYZob{}}\PY{n}{personne}\PY{p}{[}\PY{l+s+s1}{\PYZsq{}}\PY{l+s+s1}{nom}\PY{l+s+s1}{\PYZsq{}}\PY{p}{]}\PY{p}{:} \PY{n}{personne} \PY{k}{for} \PY{n}{personne} \PY{o+ow}{in} \PY{n}{personnes}\PY{p}{\PYZcb{}}
        \PY{n}{index\PYZus{}par\PYZus{}nom}
\end{Verbatim}


\begin{Verbatim}[commandchars=\\\{\}]
{\color{outcolor}Out[{\color{outcolor}4}]:} \{'Jacques': \{'age': 52, 'email': 'jacques@example.com', 'nom': 'Jacques'\},
         'Paul': \{'age': 18, 'email': 'paul@example.com', 'nom': 'Paul'\},
         'Pierre': \{'age': 26, 'email': 'pierre@example.com', 'nom': 'Pierre'\}\}
\end{Verbatim}
            
    \begin{Verbatim}[commandchars=\\\{\}]
{\color{incolor}In [{\color{incolor}5}]:} \PY{c+c1}{\PYZsh{} du coup pour accéder à l\PYZsq{}enregistrement pour Pierre}
        \PY{n}{index\PYZus{}par\PYZus{}nom}\PY{p}{[}\PY{l+s+s1}{\PYZsq{}}\PY{l+s+s1}{Pierre}\PY{l+s+s1}{\PYZsq{}}\PY{p}{]}
\end{Verbatim}


\begin{Verbatim}[commandchars=\\\{\}]
{\color{outcolor}Out[{\color{outcolor}5}]:} \{'age': 26, 'email': 'pierre@example.com', 'nom': 'Pierre'\}
\end{Verbatim}
            
    Attardons-nous un tout petit peu~; nous avons construit un dictionnaire
par compréhension, en créant autant d'entrées que de personnes. Nous
aborderons en détail la notion de compréhension de sets et de
dictionnaires en semaine 5, donc si cette notation vous paraît étrange
pour le moment, pas d'inquiétude.\\

Le résultat est donc un dictionnaire qu'on peut afficher comme ceci~:

    \begin{Verbatim}[commandchars=\\\{\}]
{\color{incolor}In [{\color{incolor}6}]:} \PY{k}{for} \PY{n}{nom}\PY{p}{,} \PY{n}{record} \PY{o+ow}{in} \PY{n}{index\PYZus{}par\PYZus{}nom}\PY{o}{.}\PY{n}{items}\PY{p}{(}\PY{p}{)}\PY{p}{:}
            \PY{n+nb}{print}\PY{p}{(}\PY{n}{f}\PY{l+s+s2}{\PYZdq{}}\PY{l+s+s2}{Nom : }\PY{l+s+si}{\PYZob{}nom\PYZcb{}}\PY{l+s+s2}{ \PYZhy{}\PYZgt{} enregistrement : }\PY{l+s+si}{\PYZob{}record\PYZcb{}}\PY{l+s+s2}{\PYZdq{}}\PY{p}{)}
\end{Verbatim}


    \begin{Verbatim}[commandchars=\\\{\}]
Nom : Pierre -> enregistrement : \{'nom': 'Pierre', 'age': 26, 'email': 'pierre@example.com'\}
Nom : Paul -> enregistrement : \{'nom': 'Paul', 'age': 18, 'email': 'paul@example.com'\}
Nom : Jacques -> enregistrement : \{'nom': 'Jacques', 'age': 52, 'email': 'jacques@example.com'\}

    \end{Verbatim}

    Dans cet exemple, le premier niveau de dictionnaire permet de trouver
rapidement un objet à partir d'un nom~; dans le second niveau au
contraire on utilise le dictionnaire pour implémenter un enregistrement,
à la façon d'un \texttt{struct} en C.

    \hypertarget{techniques-similaires}{%
\subsubsection{Techniques similaires}\label{techniques-similaires}}

    Notons enfin qu'il existe aussi, en Python, un autre mécanisme qui peut
être utilisé pour gérer ce genre d'objets composites, ce sont les
classes que nous verrons en semaine 6, et qui permettent de définir de
nouveaux \texttt{types} plutôt que, comme nous l'avons fait ici,
d'utiliser un type prédéfini. Dans ce sens, l'utilisation d'une classe
permet davantage de souplesse, au prix de davantage d'effort.

    \hypertarget{compluxe9ment---niveau-avancuxe9}{%
\subsection{Complément - niveau
avancé}\label{compluxe9ment---niveau-avancuxe9}}

    \hypertarget{la-muxeame-iduxe9e-mais-avec-une-classe-personne}{%
\paragraph{\texorpdfstring{La même idée, mais avec une classe
\texttt{Personne}}{La même idée, mais avec une classe Personne}\\\\}\label{la-muxeame-iduxe9e-mais-avec-une-classe-personne}}

    Je vais donner ici une implémentation du code ci-dessus, qui utilise une
classe pour modéliser les personnes. Naturellement je n'entre pas dans
les détails, que l'on verra en semaine 6, mais j'espère vous donner un
aperçu des classes dans un usage réaliste, et vous montrer les avantages
de cette approche.\\

    Pour commencer je définis la classe \texttt{Personne}, qui va me servir
à modéliser chaque personne~:

    \begin{Verbatim}[commandchars=\\\{\}]
{\color{incolor}In [{\color{incolor}7}]:} \PY{k}{class} \PY{n+nc}{Personne}\PY{p}{:}
        
            \PY{c+c1}{\PYZsh{} le constructeur \PYZhy{} vous ignorez le paramètre self,}
            \PY{c+c1}{\PYZsh{} on pourra construire une personne à partir de}
            \PY{c+c1}{\PYZsh{} 3 paramètres}
            \PY{k}{def} \PY{n+nf}{\PYZus{}\PYZus{}init\PYZus{}\PYZus{}}\PY{p}{(}\PY{n+nb+bp}{self}\PY{p}{,} \PY{n}{nom}\PY{p}{,} \PY{n}{age}\PY{p}{,} \PY{n}{email}\PY{p}{)}\PY{p}{:}
                \PY{n+nb+bp}{self}\PY{o}{.}\PY{n}{nom} \PY{o}{=} \PY{n}{nom}
                \PY{n+nb+bp}{self}\PY{o}{.}\PY{n}{age} \PY{o}{=} \PY{n}{age}
                \PY{n+nb+bp}{self}\PY{o}{.}\PY{n}{email} \PY{o}{=} \PY{n}{email}
        
            \PY{c+c1}{\PYZsh{} je définis cette méthode pour avoir}
            \PY{c+c1}{\PYZsh{} quelque chose de lisible quand je print()}
            \PY{k}{def} \PY{n+nf}{\PYZus{}\PYZus{}repr\PYZus{}\PYZus{}}\PY{p}{(}\PY{n+nb+bp}{self}\PY{p}{)}\PY{p}{:}
                \PY{k}{return} \PY{n}{f}\PY{l+s+s2}{\PYZdq{}}\PY{l+s+si}{\PYZob{}self.nom\PYZcb{}}\PY{l+s+s2}{ (}\PY{l+s+si}{\PYZob{}self.age\PYZcb{}}\PY{l+s+s2}{ ans) sur }\PY{l+s+si}{\PYZob{}self.email\PYZcb{}}\PY{l+s+s2}{\PYZdq{}}
\end{Verbatim}


    Pour construire ma liste de personnes, je fais alors~:

    \begin{Verbatim}[commandchars=\\\{\}]
{\color{incolor}In [{\color{incolor}8}]:} \PY{n}{personnes2} \PY{o}{=} \PY{p}{[}
            \PY{n}{Personne}\PY{p}{(}\PY{l+s+s1}{\PYZsq{}}\PY{l+s+s1}{Pierre}\PY{l+s+s1}{\PYZsq{}}\PY{p}{,}  \PY{l+m+mi}{25}\PY{p}{,} \PY{l+s+s1}{\PYZsq{}}\PY{l+s+s1}{pierre@example.com}\PY{l+s+s1}{\PYZsq{}}\PY{p}{)}\PY{p}{,}
            \PY{n}{Personne}\PY{p}{(}\PY{l+s+s1}{\PYZsq{}}\PY{l+s+s1}{Paul}\PY{l+s+s1}{\PYZsq{}}\PY{p}{,}    \PY{l+m+mi}{18}\PY{p}{,} \PY{l+s+s1}{\PYZsq{}}\PY{l+s+s1}{paul@example.com}\PY{l+s+s1}{\PYZsq{}}\PY{p}{)}\PY{p}{,}
            \PY{n}{Personne}\PY{p}{(}\PY{l+s+s1}{\PYZsq{}}\PY{l+s+s1}{Jacques}\PY{l+s+s1}{\PYZsq{}}\PY{p}{,} \PY{l+m+mi}{52}\PY{p}{,} \PY{l+s+s1}{\PYZsq{}}\PY{l+s+s1}{jacques@example.com}\PY{l+s+s1}{\PYZsq{}}\PY{p}{)}\PY{p}{,}
        \PY{p}{]}
\end{Verbatim}


    Si je regarde un élément de la liste j'obtiens~:

    \begin{Verbatim}[commandchars=\\\{\}]
{\color{incolor}In [{\color{incolor}9}]:} \PY{n}{personnes2}\PY{p}{[}\PY{l+m+mi}{0}\PY{p}{]}
\end{Verbatim}


\begin{Verbatim}[commandchars=\\\{\}]
{\color{outcolor}Out[{\color{outcolor}9}]:} Pierre (25 ans) sur pierre@example.com
\end{Verbatim}
            
    Je peux indexer tout ceci comme tout à l'heure, si j'ai besoin d'un
accès rapide~:

    \begin{Verbatim}[commandchars=\\\{\}]
{\color{incolor}In [{\color{incolor}10}]:} \PY{c+c1}{\PYZsh{} je dois utiliser cette fois personne.nom et non plus personne[\PYZsq{}nom\PYZsq{}]}
         \PY{n}{index2} \PY{o}{=} \PY{p}{\PYZob{}}\PY{n}{personne}\PY{o}{.}\PY{n}{nom} \PY{p}{:} \PY{n}{personne} \PY{k}{for} \PY{n}{personne} \PY{o+ow}{in} \PY{n}{personnes2}\PY{p}{\PYZcb{}}
\end{Verbatim}


    Le principe ici est exactement identique à ce qu'on a fait avec le
dictionnaire de dictionnaires, mais on a construit un dictionnaire
d'instances.\\

Et de cette façon~:

    \begin{Verbatim}[commandchars=\\\{\}]
{\color{incolor}In [{\color{incolor}11}]:} \PY{n+nb}{print}\PY{p}{(}\PY{n}{index2}\PY{p}{[}\PY{l+s+s1}{\PYZsq{}}\PY{l+s+s1}{Pierre}\PY{l+s+s1}{\PYZsq{}}\PY{p}{]}\PY{p}{)}
\end{Verbatim}


    \begin{Verbatim}[commandchars=\\\{\}]
Pierre (25 ans) sur pierre@example.com

    \end{Verbatim}

    Rendez-vous en semaine 6 pour approfondir la notion de classes et
d'instances.