    \hypertarget{try-else-finally}{%
\section{\texorpdfstring{\texttt{try} \ldots{} \texttt{else} \ldots{}
\texttt{finally}}{try \ldots{} else \ldots{} finally}}\label{try-else-finally}}

    \hypertarget{compluxe9ment---niveau-intermuxe9diaire}{%
\subsection{Complément - niveau
intermédiaire}\label{compluxe9ment---niveau-intermuxe9diaire}}

    L'instruction \texttt{try} est généralement assortie d'une une ou
plusieurs clauses \texttt{except}, comme on l'a vu dans la vidéo.\\

Sachez que l'on peut aussi utiliser - après toutes les clauses
\texttt{except}~:

\begin{itemize}
	\item 
	une clause \texttt{else}, qui va être exécutée si
	aucune exception n'est attrapée~;
	\item 
	t/ou une clause \texttt{finally} qui sera alors exécutée quoi qu'il arrive.
\end{itemize}

    Voyons cela sur des exemples.

    \hypertarget{finally}{%
\subsubsection{\texorpdfstring{\texttt{finally}}{finally}}\label{finally}}

    C'est sans doute \texttt{finally} qui est la plus utile de ces deux
clauses, car elle permet de faire un nettoyage \textbf{dans tous les cas
de figure} - de ce point de vue, cela rappelle un peu les \emph{context
managers}.\\

Et par exemple, comme avec les \emph{context managers}, une fonction
peut faire des choses même après un \texttt{return}.

    \begin{Verbatim}[commandchars=\\\{\}]
{\color{incolor}In [{\color{incolor}1}]:} \PY{c+c1}{\PYZsh{} une fonction qui fait des choses après un return}
        \PY{k}{def} \PY{n+nf}{return\PYZus{}with\PYZus{}finally}\PY{p}{(}\PY{n}{number}\PY{p}{)}\PY{p}{:}
            \PY{k}{try}\PY{p}{:}
                \PY{k}{return} \PY{l+m+mi}{1}\PY{o}{/}\PY{n}{number}
            \PY{k}{except} \PY{n+ne}{ZeroDivisionError} \PY{k}{as} \PY{n}{e}\PY{p}{:}
                \PY{n+nb}{print}\PY{p}{(}\PY{n}{f}\PY{l+s+s2}{\PYZdq{}}\PY{l+s+s2}{OOPS, }\PY{l+s+s2}{\PYZob{}}\PY{l+s+s2}{type(e)\PYZcb{}, }\PY{l+s+si}{\PYZob{}e\PYZcb{}}\PY{l+s+s2}{\PYZdq{}}\PY{p}{)}
                \PY{k}{return}\PY{p}{(}\PY{l+s+s2}{\PYZdq{}}\PY{l+s+s2}{zero\PYZhy{}divide}\PY{l+s+s2}{\PYZdq{}}\PY{p}{)}
            \PY{k}{finally}\PY{p}{:}
                \PY{n+nb}{print}\PY{p}{(}\PY{l+s+s2}{\PYZdq{}}\PY{l+s+s2}{on passe ici même si on a vu un return}\PY{l+s+s2}{\PYZdq{}}\PY{p}{)}
\end{Verbatim}


    \begin{Verbatim}[commandchars=\\\{\}]
{\color{incolor}In [{\color{incolor}2}]:} \PY{c+c1}{\PYZsh{} sans exception}
        \PY{n}{return\PYZus{}with\PYZus{}finally}\PY{p}{(}\PY{l+m+mi}{1}\PY{p}{)}
\end{Verbatim}


    \begin{Verbatim}[commandchars=\\\{\}]
on passe ici même si on a vu un return

    \end{Verbatim}

\begin{Verbatim}[commandchars=\\\{\}]
{\color{outcolor}Out[{\color{outcolor}2}]:} 1.0
\end{Verbatim}
            
    \begin{Verbatim}[commandchars=\\\{\}]
{\color{incolor}In [{\color{incolor}3}]:} \PY{c+c1}{\PYZsh{} avec exception}
        \PY{n}{return\PYZus{}with\PYZus{}finally}\PY{p}{(}\PY{l+m+mi}{0}\PY{p}{)}
\end{Verbatim}


    \begin{Verbatim}[commandchars=\\\{\}]
OOPS, <class 'ZeroDivisionError'>, division by zero
on passe ici même si on a vu un return

    \end{Verbatim}

\begin{Verbatim}[commandchars=\\\{\}]
{\color{outcolor}Out[{\color{outcolor}3}]:} 'zero-divide'
\end{Verbatim}
            
    \hypertarget{else}{%
\subsubsection{\texorpdfstring{\texttt{else}}{else}}\label{else}}

    La logique ici est assez similaire, sauf que le code du \texttt{else}
n'est exécuté que dans le cas où aucune exception n'est attrapée.\\

    En première approximation, on pourrait penser que c'est équivalent de
mettre du code dans la clause \texttt{else} ou à la fin de la clause
\texttt{try}. En fait il y a une différence subtile~:\\

\begin{quote}
\emph{The use of the \texttt{else} clause is better than adding
additional code to the \texttt{try} clause because it avoids
accidentally catching an exception that wasn't raised by the code being
protected by the \texttt{try} \ldots{} \texttt{except} statement.}
\end{quote}

Dit autrement, si le code dans la clause \texttt{else} lève une
exception, celle-ci ne \textbf{sera pas attrapée} par le \texttt{try}
courant, et sera donc propagée.\\

    Voici un exemple rapide, en pratique on rencontre assez peu souvent une
clause \texttt{else} dans un \texttt{try}~:

    \begin{Verbatim}[commandchars=\\\{\}]
{\color{incolor}In [{\color{incolor}4}]:} \PY{c+c1}{\PYZsh{} pour montrer la clause else dans un usage banal}
        \PY{k}{def} \PY{n+nf}{function\PYZus{}with\PYZus{}else}\PY{p}{(}\PY{n}{number}\PY{p}{)}\PY{p}{:}
            \PY{k}{try}\PY{p}{:}
                \PY{n}{x} \PY{o}{=} \PY{l+m+mi}{1}\PY{o}{/}\PY{n}{number}
            \PY{k}{except} \PY{n+ne}{ZeroDivisionError} \PY{k}{as} \PY{n}{e}\PY{p}{:}
                \PY{n+nb}{print}\PY{p}{(}\PY{n}{f}\PY{l+s+s2}{\PYZdq{}}\PY{l+s+s2}{OOPS, }\PY{l+s+s2}{\PYZob{}}\PY{l+s+s2}{type(e)\PYZcb{}, }\PY{l+s+si}{\PYZob{}e\PYZcb{}}\PY{l+s+s2}{\PYZdq{}}\PY{p}{)}
            \PY{k}{else}\PY{p}{:}
                \PY{n+nb}{print}\PY{p}{(}\PY{l+s+s2}{\PYZdq{}}\PY{l+s+s2}{on passe ici seulement avec un nombre non nul}\PY{l+s+s2}{\PYZdq{}}\PY{p}{)}
            \PY{k}{return} \PY{l+s+s1}{\PYZsq{}}\PY{l+s+s1}{something else}\PY{l+s+s1}{\PYZsq{}}
\end{Verbatim}


    \begin{Verbatim}[commandchars=\\\{\}]
{\color{incolor}In [{\color{incolor}5}]:} \PY{c+c1}{\PYZsh{} sans exception}
        \PY{n}{function\PYZus{}with\PYZus{}else}\PY{p}{(}\PY{l+m+mi}{1}\PY{p}{)}
\end{Verbatim}


    \begin{Verbatim}[commandchars=\\\{\}]
on passe ici seulement avec un nombre non nul

    \end{Verbatim}

\begin{Verbatim}[commandchars=\\\{\}]
{\color{outcolor}Out[{\color{outcolor}5}]:} 'something else'
\end{Verbatim}
            
    \begin{Verbatim}[commandchars=\\\{\}]
{\color{incolor}In [{\color{incolor}6}]:} \PY{c+c1}{\PYZsh{} avec exception}
        \PY{n}{function\PYZus{}with\PYZus{}else}\PY{p}{(}\PY{l+m+mi}{0}\PY{p}{)}
\end{Verbatim}


    \begin{Verbatim}[commandchars=\\\{\}]
OOPS, <class 'ZeroDivisionError'>, division by zero

    \end{Verbatim}

\begin{Verbatim}[commandchars=\\\{\}]
{\color{outcolor}Out[{\color{outcolor}6}]:} 'something else'
\end{Verbatim}
            
    Remarquez que \texttt{else} ne présente pas cette particularité de
``traverser'' le \texttt{return}, que l'on a vue avec \texttt{finally}~:

    \begin{Verbatim}[commandchars=\\\{\}]
{\color{incolor}In [{\color{incolor}7}]:} \PY{c+c1}{\PYZsh{} la clause else ne traverse pas les return}
        \PY{k}{def} \PY{n+nf}{return\PYZus{}with\PYZus{}else}\PY{p}{(}\PY{n}{number}\PY{p}{)}\PY{p}{:}
            \PY{k}{try}\PY{p}{:}
                \PY{k}{return} \PY{l+m+mi}{1}\PY{o}{/}\PY{n}{number}
            \PY{k}{except} \PY{n+ne}{ZeroDivisionError} \PY{k}{as} \PY{n}{e}\PY{p}{:}
                \PY{n+nb}{print}\PY{p}{(}\PY{n}{f}\PY{l+s+s2}{\PYZdq{}}\PY{l+s+s2}{OOPS, }\PY{l+s+s2}{\PYZob{}}\PY{l+s+s2}{type(e)\PYZcb{}, }\PY{l+s+si}{\PYZob{}e\PYZcb{}}\PY{l+s+s2}{\PYZdq{}}\PY{p}{)}
                \PY{k}{return}\PY{p}{(}\PY{l+s+s2}{\PYZdq{}}\PY{l+s+s2}{zero\PYZhy{}divide}\PY{l+s+s2}{\PYZdq{}}\PY{p}{)}
            \PY{k}{else}\PY{p}{:}
                \PY{n+nb}{print}\PY{p}{(}\PY{l+s+s2}{\PYZdq{}}\PY{l+s+s2}{on ne passe jamais ici à cause des return}\PY{l+s+s2}{\PYZdq{}}\PY{p}{)}
\end{Verbatim}


    \begin{Verbatim}[commandchars=\\\{\}]
{\color{incolor}In [{\color{incolor}8}]:} \PY{c+c1}{\PYZsh{} sans exception}
        \PY{n}{return\PYZus{}with\PYZus{}else}\PY{p}{(}\PY{l+m+mi}{1}\PY{p}{)}
\end{Verbatim}


\begin{Verbatim}[commandchars=\\\{\}]
{\color{outcolor}Out[{\color{outcolor}8}]:} 1.0
\end{Verbatim}
            
    \begin{Verbatim}[commandchars=\\\{\}]
{\color{incolor}In [{\color{incolor}9}]:} \PY{c+c1}{\PYZsh{} avec exception}
        \PY{n}{return\PYZus{}with\PYZus{}else}\PY{p}{(}\PY{l+m+mi}{0}\PY{p}{)}
\end{Verbatim}


    \begin{Verbatim}[commandchars=\\\{\}]
OOPS, <class 'ZeroDivisionError'>, division by zero

    \end{Verbatim}

\begin{Verbatim}[commandchars=\\\{\}]
{\color{outcolor}Out[{\color{outcolor}9}]:} 'zero-divide'
\end{Verbatim}
            
    \hypertarget{pour-en-savoir-plus}{%
\subsubsection{Pour en savoir plus}\label{pour-en-savoir-plus}}

    Voyez
\href{https://docs.python.org/3/tutorial/errors.html\#handling-exceptions}{le
tutorial sur les exceptions} dans la documentation officielle.