    \hypertarget{formats-de-fichiers-json-et-autres}{%
\section{Formats de fichiers~: JSON et
autres}\label{formats-de-fichiers-json-et-autres}}

    \hypertarget{compluxe9ments---niveau-basique}{%
\subsection{Compléments - niveau
basique}\label{compluxe9ments---niveau-basique}}

    Voici quelques mots sur des outils Python fournis dans la bibliothèque
standard, et qui permettent de lire ou écrire des données dans des
fichiers.

    \hypertarget{le-probluxe8me}{%
\subsubsection{Le problème}\label{le-probluxe8me}}

    Les données dans un programme Python sont stockées en mémoire (la RAM),
sous une forme propice aux calculs. Par exemple un petit entier est
fréquemment stocké en binaire dans un mot de 64 bits, qui est prêt à
être soumis au processeur pour faire une opération arithmétique.\\

    Ce format ne se prête pas forcément toujours à être transposé tel quel
lorsqu'on doit écrire des données sur un support plus pérenne, comme un
disque dur, ou encore sur un réseau pour transmission distante - ces
deux supports étant à ce point de vue très voisins.\\

Ainsi par exemple il pourra être plus commode d'écrire notre entier sur
disque, ou de le transmettre à un programme distant, sous une forme
décimale qui sera plus lisible, sachant que par ailleurs toutes les
machines ne codent pas un entier de la même façon.\\

    Il convient donc de faire de la traduction dans les deux sens entre
représentations d'une part en mémoire, et d'autre part sur disque ou sur
réseau (à nouveau, on utilise en général les mêmes formats pour ces deux
usages).

    \hypertarget{le-format-json}{%
\subsubsection{Le format JSON}\label{le-format-json}}

    Le format sans aucun doute le plus populaire à l'heure actuelle est
\href{http://fr.wikipedia.org/wiki/JavaScript_Object_Notation}{le format
JSON} pour \emph{JavaScript Object Notation}.\\

Sans trop nous attarder nous dirons que JSON est un encodage - en
anglais
\href{http://en.wikipedia.org/wiki/Marshalling_\%28computer_science\%29}{marshalling}
- qui se prête bien à la plupart des types de base que l'on trouve dans
les langages modernes comme Python, Ruby ou JavaScript.\\

La bibliothèque standard de Python contient
\href{https://docs.python.org/3/library/json.html}{le module json} que
nous illustrons très rapidement ici~:

    \begin{Verbatim}[commandchars=\\\{\}]
{\color{incolor}In [{\color{incolor}1}]:} \PY{k+kn}{import} \PY{n+nn}{json}
        
        \PY{c+c1}{\PYZsh{} En partant d\PYZsq{}une donnée construite à partir de types de base}
        \PY{n}{data} \PY{o}{=} \PY{p}{[}
            \PY{c+c1}{\PYZsh{} des types qui ne posent pas de problème}
            \PY{p}{[}\PY{l+m+mi}{1}\PY{p}{,} \PY{l+m+mi}{2}\PY{p}{,} \PY{l+s+s1}{\PYZsq{}}\PY{l+s+s1}{a}\PY{l+s+s1}{\PYZsq{}}\PY{p}{,} \PY{p}{[}\PY{l+m+mf}{3.23}\PY{p}{,} \PY{l+m+mf}{4.32}\PY{p}{]}\PY{p}{,} \PY{p}{\PYZob{}}\PY{l+s+s1}{\PYZsq{}}\PY{l+s+s1}{eric}\PY{l+s+s1}{\PYZsq{}}\PY{p}{:} \PY{l+m+mi}{32}\PY{p}{,} \PY{l+s+s1}{\PYZsq{}}\PY{l+s+s1}{jean}\PY{l+s+s1}{\PYZsq{}}\PY{p}{:} \PY{l+m+mi}{43}\PY{p}{\PYZcb{}}\PY{p}{]}\PY{p}{,}
            \PY{c+c1}{\PYZsh{} un tuple}
            \PY{p}{(}\PY{l+m+mi}{1}\PY{p}{,} \PY{l+m+mi}{2}\PY{p}{,} \PY{l+m+mi}{3}\PY{p}{)}\PY{p}{,}
        \PY{p}{]}
        
        \PY{c+c1}{\PYZsh{} sauver ceci dans un fichier}
        \PY{k}{with} \PY{n+nb}{open}\PY{p}{(}\PY{l+s+s2}{\PYZdq{}}\PY{l+s+s2}{s1.json}\PY{l+s+s2}{\PYZdq{}}\PY{p}{,}\PY{l+s+s2}{\PYZdq{}}\PY{l+s+s2}{w}\PY{l+s+s2}{\PYZdq{}}\PY{p}{,} \PY{n}{encoding}\PY{o}{=}\PY{l+s+s1}{\PYZsq{}}\PY{l+s+s1}{utf\PYZhy{}8}\PY{l+s+s1}{\PYZsq{}}\PY{p}{)} \PY{k}{as} \PY{n}{json\PYZus{}output}\PY{p}{:}
            \PY{n}{json}\PY{o}{.}\PY{n}{dump}\PY{p}{(}\PY{n}{data}\PY{p}{,} \PY{n}{json\PYZus{}output}\PY{p}{)}
        
        \PY{c+c1}{\PYZsh{} et relire le résultat}
        \PY{k}{with} \PY{n+nb}{open}\PY{p}{(}\PY{l+s+s2}{\PYZdq{}}\PY{l+s+s2}{s1.json}\PY{l+s+s2}{\PYZdq{}}\PY{p}{,} \PY{n}{encoding}\PY{o}{=}\PY{l+s+s1}{\PYZsq{}}\PY{l+s+s1}{utf\PYZhy{}8}\PY{l+s+s1}{\PYZsq{}}\PY{p}{)} \PY{k}{as} \PY{n}{json\PYZus{}input}\PY{p}{:}
            \PY{n}{data2} \PY{o}{=} \PY{n}{json}\PY{o}{.}\PY{n}{load}\PY{p}{(}\PY{n}{json\PYZus{}input}\PY{p}{)}
\end{Verbatim}


    \hypertarget{limitations-de-json}{%
\subparagraph{Limitations de json\\\\}\label{limitations-de-json}}

    Certains types de base ne sont pas supportés par le format JSON (car ils
ne sont pas natifs en JavaScript), c'est le cas notamment pour~:

\begin{itemize}
	\item 
	\texttt{tuple}, qui se fait encoder comme une liste~;
	\item
	\texttt{complex}, \texttt{set} et \texttt{frozenset}, que l'on ne peut
	pas encoder du tout (sans étendre la bibliothèque).
\end{itemize}

    C'est ce qui explique ce qui suit~:

    \begin{Verbatim}[commandchars=\\\{\}]
{\color{incolor}In [{\color{incolor}2}]:} \PY{c+c1}{\PYZsh{} le premier élément de data est intact,}
        \PY{c+c1}{\PYZsh{} comme si on avait fait une *deep copy* en fait}
        \PY{n+nb}{print}\PY{p}{(}\PY{l+s+s2}{\PYZdq{}}\PY{l+s+s2}{première partie de data}\PY{l+s+s2}{\PYZdq{}}\PY{p}{,} \PY{n}{data}\PY{p}{[}\PY{l+m+mi}{0}\PY{p}{]} \PY{o}{==} \PY{n}{data2}\PY{p}{[}\PY{l+m+mi}{0}\PY{p}{]}\PY{p}{)}
\end{Verbatim}


    \begin{Verbatim}[commandchars=\\\{\}]
première partie de data True

    \end{Verbatim}

    \begin{Verbatim}[commandchars=\\\{\}]
{\color{incolor}In [{\color{incolor}3}]:} \PY{c+c1}{\PYZsh{} par contre le `tuple` se fait encoder comme une `list`}
        \PY{n+nb}{print}\PY{p}{(}\PY{l+s+s2}{\PYZdq{}}\PY{l+s+s2}{deuxième partie}\PY{l+s+s2}{\PYZdq{}}\PY{p}{,} \PY{l+s+s2}{\PYZdq{}}\PY{l+s+s2}{entrée}\PY{l+s+s2}{\PYZdq{}}\PY{p}{,} \PY{n+nb}{type}\PY{p}{(}\PY{n}{data}\PY{p}{[}\PY{l+m+mi}{1}\PY{p}{]}\PY{p}{)}\PY{p}{,} \PY{l+s+s2}{\PYZdq{}}\PY{l+s+s2}{sortie}\PY{l+s+s2}{\PYZdq{}}\PY{p}{,} \PY{n+nb}{type}\PY{p}{(}\PY{n}{data2}\PY{p}{[}\PY{l+m+mi}{1}\PY{p}{]}\PY{p}{)}\PY{p}{)}
\end{Verbatim}


    \begin{Verbatim}[commandchars=\\\{\}]
deuxième partie entrée <class 'tuple'> sortie <class 'list'>

    \end{Verbatim}

    Malgré ces petites limitations, ce format est de plus en plus populaire,
notamment parce qu'on peut l'utiliser pour communiquer avec des
applications Web écrites en JavaScript, et aussi parce qu'il est très
léger, et supporté par de nombreux langages.

    \hypertarget{compluxe9ments---niveau-intermuxe9diaire}{%
\subsection{Compléments - niveau
intermédiaire}\label{compluxe9ments---niveau-intermuxe9diaire}}

    \hypertarget{le-format-csv}{%
\subsubsection{\texorpdfstring{Le format
\texttt{csv}}{Le format csv}}\label{le-format-csv}}

    Le format \texttt{csv} pour \emph{Comma Separated Values}, originaire du
monde des tableurs, peut rendre service à l'occasion, il est proposé
\href{https://docs.python.org/3/library/csv.html}{dans le module
\texttt{csv}}.

    \hypertarget{le-format-pickle}{%
\subsubsection{Le format pickle}\label{le-format-pickle}}

    Le format \texttt{pickle} remplit une fonctionnalité très voisine de
\texttt{JSON}, mais est spécifique à Python. C'est pourquoi, malgré des
limites un peu moins sévères, son usage tend à rester plutôt marginal
pour l'échange de données, on lui préfère en général le format JSON.\\

Par contre, pour la sauvegarde locale d'objets Python (pour, par
exemple, faire des points de reprises d'un programme), il est très
utile. Il est implémenté
\href{https://docs.python.org/3/library/pickle.html}{dans le module
\texttt{pickle}}.

    \hypertarget{le-format-xml}{%
\subsubsection{Le format XML}\label{le-format-xml}}

    Vous avez aussi très probablement entendu parler de XML, qui est un
format assez populaire également.\\

Cela dit, la puissance, et donc le coût, de XML et JSON ne sont pas du
tout comparables, XML étant beaucoup plus flexible mais au prix d'une
complexité de mise en œuvre très supérieure.\\

Il existe plusieurs souches différentes de bibliothèques prenant en
charge le format XML,
\href{https://docs.python.org/3/library/xml.html}{qui sont introduites
ici}.

    \hypertarget{pour-en-savoir-plus}{%
\subsubsection{Pour en savoir plus}\label{pour-en-savoir-plus}}

    Voyez la page sur
\href{https://docs.python.org/3/library/fileformats.html}{les formats de
fichiers} dans la documentation Python.