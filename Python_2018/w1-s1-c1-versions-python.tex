    \hypertarget{versions-de-python}{%
\section{Versions de Python}\label{versions-de-python}}

    \hypertarget{version-de-ruxe9fuxe9rence-python-3.6}{%
\subsubsection{Version de référence:
Python-3.6}\label{version-de-ruxe9fuxe9rence-python-3.6}}

    Comme on l'indique dans la vidéo, la version de Python qui a servi de
\textbf{référence pour le MOOC est la version 3.6}, c'est notamment avec
cette version que l'on a tourné les vidéos.

    \hypertarget{versions-plus-anciennes}{%
\subsubsection{Versions plus anciennes}\label{versions-plus-anciennes}}

    Certaines précautions sont à prendre si vous utilisez une version plus
ancienne:

    \hypertarget{python-3.5}{%
\subparagraph{Python-3.5}\label{python-3.5}}

    Si vous préférez utiliser python-3.5, la différence la plus visible pour
vous apparaitra avec les \emph{f-strings}~:

    \begin{Verbatim}[commandchars=\\\{\}]
{\color{incolor}In [{\color{incolor}1}]:} \PY{n}{age} \PY{o}{=} \PY{l+m+mi}{10}
        
        \PY{c+c1}{\PYZsh{} un exemple de f\PYZhy{}string}
        \PY{n}{f}\PY{l+s+s2}{\PYZdq{}}\PY{l+s+s2}{Jean a }\PY{l+s+si}{\PYZob{}age\PYZcb{}}\PY{l+s+s2}{ ans}\PY{l+s+s2}{\PYZdq{}}
\end{Verbatim}


\begin{Verbatim}[commandchars=\\\{\}]
{\color{outcolor}Out[{\color{outcolor}1}]:} 'Jean a 10 ans'
\end{Verbatim}
            
    Cette construction - que nous utilisons très fréquemment - n'a été
introduite qu'en Python-3.6, aussi si vous utilisez Python-3.5 vous
verrez ceci~:

\begin{Shaded}
\begin{Highlighting}[]
\OperatorTok{>>>}\NormalTok{ age }\OperatorTok{=} \DecValTok{10}
\OperatorTok{>>>} \SpecialStringTok{f"Jean a }\SpecialCharTok{\{}\NormalTok{age}\SpecialCharTok{\}}\SpecialStringTok{ ans"}
\NormalTok{  File }\StringTok{"<stdin>"}\NormalTok{, line }\DecValTok{1}
    \SpecialStringTok{f"Jean a }\SpecialCharTok{\{}\NormalTok{age}\SpecialCharTok{\}}\SpecialStringTok{ ans"}
                      \OperatorTok{^}
\PreprocessorTok{SyntaxError}\NormalTok{: invalid syntax}
\end{Highlighting}
\end{Shaded}

    Dans ce cas vous devrez remplacer ce code avec la méthode
\texttt{format} - que nous verrons en Semaine 2 avec les chaines de
caractères - et dans le cas présent il faudrait remplacer par ceci~:

    \begin{Verbatim}[commandchars=\\\{\}]
{\color{incolor}In [{\color{incolor}2}]:} \PY{n}{age} \PY{o}{=} \PY{l+m+mi}{10}
        
        \PY{l+s+s2}{\PYZdq{}}\PY{l+s+s2}{Jean a }\PY{l+s+si}{\PYZob{}\PYZcb{}}\PY{l+s+s2}{ ans}\PY{l+s+s2}{\PYZdq{}}\PY{o}{.}\PY{n}{format}\PY{p}{(}\PY{n}{age}\PY{p}{)}
\end{Verbatim}


\begin{Verbatim}[commandchars=\\\{\}]
{\color{outcolor}Out[{\color{outcolor}2}]:} 'Jean a 10 ans'
\end{Verbatim}
            
    Comme ces f-strings sont très présents dans le cours, il est recommandé
d'utiliser au moins python-3.6.

    \hypertarget{python-3.4}{%
\subparagraph{Python-3.4}\label{python-3.4}}

    La remarque vaut donc \emph{a fortiori} pour python-3.4 qui, en outre,
ne vous permettra pas de suivre la semaine 8 sur la programmation
asynchrone, car les mots-clés \texttt{async} et \texttt{await} ont été
introduits seulement dans Python-3.5.

    \hypertarget{version-utilisuxe9e-dans-les-notebooks-versions-plus-ruxe9centes}{%
\subsubsection{Version utilisée dans les notebooks / versions plus
récentes}\label{version-utilisuxe9e-dans-les-notebooks-versions-plus-ruxe9centes}}

    Tout le cours doit pouvoir s'exécuter tel quel avec une version plus
récente de Python.\\

Cela dit, certains compléments illustrent des nouveautés apparues après
la 3.6, comme les \emph{dataclasses} qui sont apparues avec python-3.7,
et que nous verrons en semaine 6.\\

Dans tous les cas, nous \textbf{signalons systématiquement} les
notebooks qui nécessitent une version plus récente que 3.6.\\

    Voici enfin, à toutes fins utiles, un premier fragment de code Python
qui affiche la version de Python utilisée dans tous les notebooks de ce
cours.\\

Nous reviendrons en détail sur l'utilisation des notebooks dans une
prochaine séquence, dans l'immédiat pour exécuter ce code vous pouvez~:

\begin{itemize}
	\item 
	désigner avec la souris la cellule de code; vous verrez alors
	apparaître une petite flêche à coté du mot \texttt{In}, en cliquant
	cette flêche vous exécutez le code;
	\item
	un autre méthode consiste à
	sélectionner la cellule de code avec la souris; une fois que c'est fait
	vous pouvez cliquer sur le bouton \texttt{\textgreater{}\textbar{}\ Run}
	dans la barre de menu (bleue claire) du notebook.
\end{itemize}

    \begin{Verbatim}[commandchars=\\\{\}]
{\color{incolor}In [{\color{incolor}3}]:} \PY{c+c1}{\PYZsh{} ce premier fragment de code affiche des détails sur la}
        \PY{c+c1}{\PYZsh{} version de python qui exécute tous les notebooks du cours}
        \PY{k+kn}{import} \PY{n+nn}{sys}
        \PY{n+nb}{print}\PY{p}{(}\PY{n}{sys}\PY{o}{.}\PY{n}{version\PYZus{}info}\PY{p}{)}
\end{Verbatim}


    \begin{Verbatim}[commandchars=\\\{\}]
sys.version\_info(major=3, minor=6, micro=3, releaselevel='final', serial=0)

    \end{Verbatim}

    Pas de panique si vous n'y arrivez pas, nous consacrons très bientôt une
séquence entière à l'utilisation des notebooks :)