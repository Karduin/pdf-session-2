    \hypertarget{expression-conditionnelle}{%
\section{Expression conditionnelle}\label{expression-conditionnelle}}

    \hypertarget{exercice---niveau-basique}{%
\subsection{Exercice - niveau basique}\label{exercice---niveau-basique}}

    \hypertarget{analyse-et-mise-en-forme}{%
\subsubsection{Analyse et mise en
forme}\label{analyse-et-mise-en-forme}}

    \begin{Verbatim}[commandchars=\\\{\}]
{\color{incolor}In [{\color{incolor} }]:} \PY{c+c1}{\PYZsh{} Pour charger l\PYZsq{}exercice}
        \PY{k+kn}{from} \PY{n+nn}{corrections}\PY{n+nn}{.}\PY{n+nn}{exo\PYZus{}libelle} \PY{k}{import} \PY{n}{exo\PYZus{}libelle}
\end{Verbatim}


    Un fichier contient, dans chaque ligne, des informations (champs)
séparées par des virgules. Les espaces et tabulations présentes dans la
ligne ne sont pas significatives et doivent être ignorées.\\

Dans cet exercice de niveau basique, on suppose que chaque ligne a
exactement 3 champs, qui représentent respectivement le prénom, le nom,
et le rang d'une personne dans un classement. Une fois les espaces et
tabulations ignorées, on ne fait pas de vérification sur le contenu des
3 champs.\\

On vous demande d'écrire la fonction \texttt{libelle}, qui sera appelée
pour chaque ligne du fichier. Cette fonction:

\begin{itemize}
	\item 
	prend en argument une ligne (chaîne de caractères)
	\item
	retourne une chaîne de caractères mise en forme (voir plus bas)
	\item
	ou bien retourne \texttt{None} si la ligne n'a
	pas pu être analysée, parce qu'elle ne vérifie pas les hypothèses
	ci-dessus (c'est notamment le cas si on ne trouve pas exactement les 3
	champs)
\end{itemize}

    La mise en forme consiste à retourner

\begin{verbatim}
Nom.Prenom (message)
 
\end{verbatim}

le \emph{message} étant lui-même le \emph{rang} mis en forme pour
afficher `1er', `2nd' ou `\emph{n}-ème' selon le cas. Voici quelques
exemples

    \begin{Verbatim}[commandchars=\\\{\}]
{\color{incolor}In [{\color{incolor} }]:} \PY{c+c1}{\PYZsh{} voici quelques exemples de ce qui est attendu}
        \PY{n}{exo\PYZus{}libelle}\PY{o}{.}\PY{n}{example}\PY{p}{(}\PY{p}{)}
\end{Verbatim}


    \begin{Verbatim}[commandchars=\\\{\}]
{\color{incolor}In [{\color{incolor} }]:} \PY{c+c1}{\PYZsh{} écrivez votre code ici}
        \PY{k}{def} \PY{n+nf}{libelle}\PY{p}{(}\PY{n}{ligne}\PY{p}{)}\PY{p}{:}
            \PY{l+s+s2}{\PYZdq{}}\PY{l+s+s2}{\PYZlt{}votre\PYZus{}code\PYZgt{}}\PY{l+s+s2}{\PYZdq{}}
\end{Verbatim}


    \begin{Verbatim}[commandchars=\\\{\}]
{\color{incolor}In [{\color{incolor} }]:} \PY{c+c1}{\PYZsh{} pour le vérifier}
        \PY{n}{exo\PYZus{}libelle}\PY{o}{.}\PY{n}{correction}\PY{p}{(}\PY{n}{libelle}\PY{p}{)}
\end{Verbatim}