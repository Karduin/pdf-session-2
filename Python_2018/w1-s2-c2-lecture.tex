
    \hypertarget{un-peu-de-lecture}{%
\section{Un peu de lecture}\label{un-peu-de-lecture}}

    \hypertarget{compluxe9ment---niveau-basique}{%
\subsection{Complément - niveau
basique}\label{compluxe9ment---niveau-basique}}

    \hypertarget{mise-uxe0-jour-de-juillet-2018}{%
\subsubsection{Mise à jour de Juillet
2018}\label{mise-uxe0-jour-de-juillet-2018}}

    Le 12 Juillet 2018, Guido van Rossum
\href{https://lwn.net/Articles/759654/}{a annoncé qu'il quittait la
fonction de BDFL} qu'il occupait depuis près de trois décennies. Il
n'est pas tout à fait clair à ce stade comment va évoluer la gouvernance
de Python.

    \hypertarget{le-zen-de-python}{%
\subsubsection{Le Zen de Python}\label{le-zen-de-python}}

    Vous pouvez lire le ``Zen de Python'', qui résume la philosophie du
langage, en important le module \texttt{this} avec ce code~: (pour
exécuter ce code, cliquez dans la cellule de code, et faites au clavier
``Majuscule/Entrée'' ou ``Shift/Enter'')

    \begin{Verbatim}[commandchars=\\\{\}]
{\color{incolor}In [{\color{incolor}1}]:} \PY{c+c1}{\PYZsh{} le Zen de Python}
        \PY{k+kn}{import} \PY{n+nn}{this}
\end{Verbatim}


    \begin{Verbatim}[commandchars=\\\{\}]
The Zen of Python, by Tim Peters

Beautiful is better than ugly.
Explicit is better than implicit.
Simple is better than complex.
Complex is better than complicated.
Flat is better than nested.
Sparse is better than dense.
Readability counts.
Special cases aren't special enough to break the rules.
Although practicality beats purity.
Errors should never pass silently.
Unless explicitly silenced.
In the face of ambiguity, refuse the temptation to guess.
There should be one-- and preferably only one --obvious way to do it.
Although that way may not be obvious at first unless you're Dutch.
Now is better than never.
Although never is often better than *right* now.
If the implementation is hard to explain, it's a bad idea.
If the implementation is easy to explain, it may be a good idea.
Namespaces are one honking great idea -- let's do more of those!

    \end{Verbatim}

    \hypertarget{documentation}{%
\subsubsection{Documentation}\label{documentation}}

    \begin{itemize}
\item
  On peut commencer par citer
  l'\href{http://fr.wikipedia.org/wiki/Python_\%28langage\%29}{article
  de Wikipédia sur Python en français}.
\item
  La \href{https://wiki.python.org/moin/FrenchLanguage}{page sur le
  langage en français}.
\item
  La \href{https://docs.python.org/3/}{documentation originale} de
  Python 3 - donc, en anglais - est un très bon point d'entrée lorsqu'on
  cherche un sujet particulier, mais (beaucoup) trop abondante pour être
  lue d'un seul trait. Pour chercher de la documentation sur un module
  particulier, le plus simple est encore d'utiliser Google - ou votre
  moteur de recherche favori - qui vous redirigera, dans la grande
  majorité des cas, vers la page qui va bien dans, précisément, la
  documentation de Python.

  À titre d'exercice, cherchez la documentation du module
  \texttt{pathlib} en
  \href{https://www.google.fr/search?q=python+module+pathlib}{cherchant
  sur Google} les mots-clé \texttt{"python\ module\ pathlib"}.
\item
  J'aimerais vous signaler également une initiative pour
  \href{https://docs.python.org/fr/3/}{traduire la documentation
  officielle en français}.
\end{itemize}

    \hypertarget{historique-et-survol}{%
\subsubsection{Historique et survol}\label{historique-et-survol}}

    \begin{itemize}
\tightlist
\item
  La FAQ officielle de Python (en anglais) sur
  \href{https://docs.python.org/3/faq/design.html}{les choix de
  conception et l'historique du langage}.
\item
  L'article de Wikipédia (en anglais) sur
  l'\href{http://en.wikipedia.org/wiki/History_of_Python}{historique du
  langage}.
\item
  Sur Wikipédia, un article (en anglais) sur
  \href{http://en.wikipedia.org/wiki/Python_syntax_and_semantics}{la
  syntaxe et la sémantique de Python}.
\end{itemize}

    \hypertarget{un-peu-de-folklore}{%
\subsubsection{Un peu de folklore}\label{un-peu-de-folklore}}

    \begin{itemize}
\tightlist
\item
  Le \href{https://www.youtube.com/watch?v=YgtL4S7Hrwo}{discours de
  Guido van Rossum à PyCon 2016}.
\item
  Sur YouTube, le
  \href{https://www.youtube.com/watch?v=anwy2MPT5RE}{sketch des Monty
  Python} d'où proviennent les termes \texttt{spam}, \texttt{eggs} et
  autres \texttt{beans} que l'on utilise traditionnellement dans les
  exemples en Python plutôt que \texttt{foo} et \texttt{bar}.
\item
  L'\href{http://en.wikipedia.org/wiki/Spam_\%28Monty_Python\%29}{article
  Wikipédia correspondant}, qui cite le langage Python.
\end{itemize}

    \hypertarget{compluxe9ment---niveau-intermuxe9diaire}{%
\subsection{Complément - niveau
intermédiaire}\label{compluxe9ment---niveau-intermuxe9diaire}}

    \hypertarget{licence}{%
\subsubsection{Licence}\label{licence}}

    \begin{itemize}
\tightlist
\item
  La \href{https://docs.python.org/3/license.html}{licence d'utilisation
  est disponible ici}.
\item
  La page de la \href{https://www.python.org/psf/}{Python Software
  Foundation}, qui est une entité légale similaire à nos associations de
  1901, à but non lucratif~; elle possède les droits sur le langage.
\end{itemize}

    \hypertarget{le-processus-de-duxe9veloppement}{%
\subsubsection{Le processus de
développement}\label{le-processus-de-duxe9veloppement}}

    \begin{itemize}
\tightlist
\item
  Comment les choix d'évolution sont proposés et discutés, au travers
  des PEP (Python Enhancement Proposals)

  \begin{itemize}
  \tightlist
  \item
    \href{http://en.wikipedia.org/wiki/Python\_(programming\_language)\#Development}{http://en.wikipedia.org/wiki/Python\_(programming\_language)\#Development}
  \end{itemize}
\item
  Le premier PEP décrit en détail le cycle de vie des PEPs

  \begin{itemize}
  \tightlist
  \item
    \href{http://legacy.python.org/dev/peps/pep-0001/}{http://legacy.python.org/dev/peps/pep-0001/}
  \end{itemize}
\item
  Le PEP 8, qui préconise un style de présentation (\emph{style guide})

  \begin{itemize}
  \tightlist
  \item
    \href{http://legacy.python.org/dev/peps/pep-0008/}{http://legacy.python.org/dev/peps/pep-0008/}
  \end{itemize}
\item
  L'index de tous les PEPs

  \begin{itemize}
  \tightlist
  \item
    \href{http://legacy.python.org/dev/peps/}{http://legacy.python.org/dev/peps/}
  \end{itemize}
\end{itemize}
