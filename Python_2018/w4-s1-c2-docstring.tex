    \hypertarget{rappels-sur-docstring}{%
\section{\texorpdfstring{Rappels sur
\emph{docstring}}{Rappels sur docstring}}\label{rappels-sur-docstring}}

    \hypertarget{compluxe9ment---niveau-basique}{%
\subsection{Complément - niveau
basique}\label{compluxe9ment---niveau-basique}}

    \hypertarget{comment-documenter-une-fonction}{%
\subsubsection{Comment documenter une
fonction}\label{comment-documenter-une-fonction}}

    Pour rappel, il est recommandé de toujours documenter les fonctions en
ajoutant une chaîne comme première instruction.

    \begin{Verbatim}[commandchars=\\\{\}]
{\color{incolor}In [{\color{incolor}1}]:} \PY{k}{def} \PY{n+nf}{flatten}\PY{p}{(}\PY{n}{containers}\PY{p}{)}\PY{p}{:}
            \PY{l+s+s2}{\PYZdq{}}\PY{l+s+s2}{returns a list of the elements of the elements in containers}\PY{l+s+s2}{\PYZdq{}}
            \PY{k}{return} \PY{p}{[}\PY{n}{element} \PY{k}{for} \PY{n}{container} \PY{o+ow}{in} \PY{n}{containers} \PY{k}{for} \PY{n}{element} \PY{o+ow}{in} \PY{n}{container}\PY{p}{]}
\end{Verbatim}


    Cette information peut être consultée, soit interactivement~:

    \begin{Verbatim}[commandchars=\\\{\}]
{\color{incolor}In [{\color{incolor}2}]:} \PY{n}{help}\PY{p}{(}\PY{n}{flatten}\PY{p}{)}
\end{Verbatim}


    \begin{Verbatim}[commandchars=\\\{\}]
Help on function flatten in module \_\_main\_\_:

flatten(containers)
    returns a list of the elements of the elements in containers


    \end{Verbatim}

    Soit programmativement~:

    \begin{Verbatim}[commandchars=\\\{\}]
{\color{incolor}In [{\color{incolor}3}]:} \PY{n}{flatten}\PY{o}{.}\PY{n+nv+vm}{\PYZus{}\PYZus{}doc\PYZus{}\PYZus{}}
\end{Verbatim}


\begin{Verbatim}[commandchars=\\\{\}]
{\color{outcolor}Out[{\color{outcolor}3}]:} 'returns a list of the elements of the elements in containers'
\end{Verbatim}
            
    \hypertarget{sous-quel-format}{%
\subsubsection{Sous quel format ?}\label{sous-quel-format}}

    L'usage est d'utiliser une chaîne simple (délimitée par «~\texttt{"}~»
ou «~\texttt{\textquotesingle{}}~») lorsque le \emph{docstring} tient
sur une seule ligne, comme ci-dessus.\\

    Lorsque ce n'est pas le cas - et pour du vrai code, c'est rarement le
cas - on utilise des chaînes multi-lignes (délimitées par
«~\texttt{"""}~» ou
«~\texttt{\textquotesingle{}\textquotesingle{}\textquotesingle{}}~»).
Dans ce cas le format est très flexible, car le \emph{docstring} est
normalisé, comme on le voit sur ces deux exemples, où le rendu final est
identique~:

    \begin{Verbatim}[commandchars=\\\{\}]
{\color{incolor}In [{\color{incolor}4}]:} \PY{c+c1}{\PYZsh{} un style de docstring multi\PYZhy{}lignes}
        \PY{k}{def} \PY{n+nf}{flatten}\PY{p}{(}\PY{n}{containers}\PY{p}{)}\PY{p}{:}
            \PY{l+s+sd}{\PYZdq{}\PYZdq{}\PYZdq{}}
        \PY{l+s+sd}{provided that containers is a list (or more generally an iterable)}
        \PY{l+s+sd}{of elements that are themselves iterables, this function}
        \PY{l+s+sd}{returns a list of the items in these elements}
        \PY{l+s+sd}{    \PYZdq{}\PYZdq{}\PYZdq{}}
            \PY{k}{return} \PY{p}{[}\PY{n}{element} \PY{k}{for} \PY{n}{container} \PY{o+ow}{in} \PY{n}{containers} \PY{k}{for} \PY{n}{element} \PY{o+ow}{in} \PY{n}{container}\PY{p}{]}
        
        \PY{n}{help}\PY{p}{(}\PY{n}{flatten}\PY{p}{)}
\end{Verbatim}


    \begin{Verbatim}[commandchars=\\\{\}]
Help on function flatten in module \_\_main\_\_:

flatten(containers)
    provided that containers is a list (or more generally an iterable)
    of elements that are themselves iterables, this function
    returns a list of the items in these elements


    \end{Verbatim}

    \begin{Verbatim}[commandchars=\\\{\}]
{\color{incolor}In [{\color{incolor}5}]:} \PY{c+c1}{\PYZsh{} un autre style, qui donne le même résultat}
        \PY{k}{def} \PY{n+nf}{flatten}\PY{p}{(}\PY{n}{containers}\PY{p}{)}\PY{p}{:}
            \PY{l+s+sd}{\PYZdq{}\PYZdq{}\PYZdq{}}
        \PY{l+s+sd}{    provided that containers is a list (or more generally an iterable)}
        \PY{l+s+sd}{    of elements that are themselves iterables, this function}
        \PY{l+s+sd}{    returns a list of the items in these elements}
        \PY{l+s+sd}{    \PYZdq{}\PYZdq{}\PYZdq{}}
            \PY{k}{return} \PY{p}{[}\PY{n}{element} \PY{k}{for} \PY{n}{container} \PY{o+ow}{in} \PY{n}{containers} \PY{k}{for} \PY{n}{element} \PY{o+ow}{in} \PY{n}{container}\PY{p}{]}
        
        \PY{n}{help}\PY{p}{(}\PY{n}{flatten}\PY{p}{)}
\end{Verbatim}


    \begin{Verbatim}[commandchars=\\\{\}]
Help on function flatten in module \_\_main\_\_:

flatten(containers)
    provided that containers is a list (or more generally an iterable)
    of elements that are themselves iterables, this function
    returns a list of the items in these elements


    \end{Verbatim}

    \hypertarget{quelle-information}{%
\subsubsection{Quelle information ?}\label{quelle-information}}

    On remarquera que dans ces exemples, le \emph{docstring} ne répète pas
le nom de la fonction ou des arguments (en mots savants, sa
\emph{signature}), et que ça n'empêche pas \texttt{help} de nous
afficher cette information.\\

Le \href{http://legacy.python.org/dev/peps/pep-0257/}{PEP 257} qui donne
les conventions autour du \emph{docstring} précise bien ceci~:

    \begin{quote}
The one-line docstring should NOT be a ``signature'' reiterating the
function/method parameters (which can be obtained by introspection).
Don't do:
\end{quote}

\begin{verbatim}
def function(a, b):
    """function(a, b) -> list"""
 
\end{verbatim}

\begin{quote}
\textless{}\ldots{}\textgreater{}
\end{quote}

\begin{quote}
The preferred form for such a docstring would be something like:
\end{quote}

\begin{verbatim}
def function(a, b):
    """Do X and return a list."""
\end{verbatim}

\begin{quote}
(Of course ``Do X'' should be replaced by a useful description!)
\end{quote}

    \hypertarget{pour-en-savoir-plus}{%
\subsubsection{Pour en savoir plus}\label{pour-en-savoir-plus}}

    Vous trouverez tous les détails sur \emph{docstring} dans le
\href{http://legacy.python.org/dev/peps/pep-0257/}{PEP 257}.