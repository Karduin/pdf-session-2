    \hypertarget{formatage-des-chaines-de-caractuxe8res}{%
\section{Formatage des chaines de
caractères}\label{formatage-des-chaines-de-caractuxe8res}}

    \hypertarget{exercice---niveau-basique}{%
\subsection{Exercice - niveau basique}\label{exercice---niveau-basique}}

    \begin{Verbatim}[commandchars=\\\{\}]
{\color{incolor}In [{\color{incolor} }]:} \PY{c+c1}{\PYZsh{} charger l\PYZsq{}exercice}
        \PY{k+kn}{from} \PY{n+nn}{exo\PYZus{}label} \PY{k}{import} \PY{n}{exo\PYZus{}label}
\end{Verbatim}


    Vous devez écrire une fonction qui prend deux arguments~:

\begin{itemize}
\tightlist
\item
  une chaîne de caractères qui désigne le prénom d'un élève~;
\item
  un entier qui indique la note obtenue.
\end{itemize}

Elle devra retourner une chaîne de caractères selon que la note est

\begin{itemize}
\tightlist
\item
  \(0 \leqslant note \lt 10\)
\item
  \(10 \leqslant note \lt 16\)
\item
  \(16 \leqslant note \leqslant 20\)
\end{itemize}

comme on le voit sur les exemples~:

    \begin{Verbatim}[commandchars=\\\{\}]
{\color{incolor}In [{\color{incolor} }]:} \PY{n}{exo\PYZus{}label}\PY{o}{.}\PY{n}{example}\PY{p}{(}\PY{p}{)}
\end{Verbatim}


    \begin{Verbatim}[commandchars=\\\{\}]
{\color{incolor}In [{\color{incolor} }]:} \PY{c+c1}{\PYZsh{} à vous de jouer}
        \PY{k}{def} \PY{n+nf}{label}\PY{p}{(}\PY{n}{prenom}\PY{p}{,} \PY{n}{note}\PY{p}{)}\PY{p}{:}
            \PY{l+s+s2}{\PYZdq{}}\PY{l+s+s2}{votre code}\PY{l+s+s2}{\PYZdq{}}
\end{Verbatim}


    \begin{Verbatim}[commandchars=\\\{\}]
{\color{incolor}In [{\color{incolor} }]:} \PY{c+c1}{\PYZsh{} pour corriger}
        \PY{n}{exo\PYZus{}label}\PY{o}{.}\PY{n}{correction}\PY{p}{(}\PY{n}{label}\PY{p}{)}
\end{Verbatim}