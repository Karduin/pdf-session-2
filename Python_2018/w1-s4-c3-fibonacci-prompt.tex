    \hypertarget{la-suite-de-fibonacci}{%
\section{La suite de Fibonacci}\label{la-suite-de-fibonacci}}

    \hypertarget{compluxe9ment---niveau-basique}{%
\subsection{Complément - niveau
basique}\label{compluxe9ment---niveau-basique}}

    Voici un premier exemple de code qui tourne.\\

Nous allons commencer par le faire tourner dans ce notebook. Nous
verrons en fin de séance comment le faire fonctionner localement sur
votre ordinateur.\\

    Le but de ce programme est de calculer la
\href{https://fr.wikipedia.org/wiki/Suite_de_Fibonacci}{suite de
Fibonacci}, qui est définie comme ceci~:

\begin{itemize}
\tightlist
\item
  \(u_0 = 1\)
\item
  \(u_1 = 1\)
\item
  \(\forall n >= 2, u_n = u_{n-1} + u_{n-2}\)
\end{itemize}

Ce qui donne pour les premières valeurs~:

    \begin{longtable}[]{@{}lr@{}}
\toprule
n & fibonacci(n)\tabularnewline
\midrule
\endhead
0 & 1\tabularnewline
1 & 1\tabularnewline
2 & 2\tabularnewline
3 & 3\tabularnewline
4 & 5\tabularnewline
5 & 8\tabularnewline
6 & 13\tabularnewline
\bottomrule
\end{longtable}

    On commence par définir la fonction \texttt{fibonacci} comme il suit.
Naturellement vous n'avez pas encore tout le bagage pour lire ce code,
ne vous inquiétez pas, nous allons vous expliquer tout ça dans les
prochaines semaines. Le but est uniquement de vous montrer un
fonctionnement de l'interpréteur Python et de IDLE.

    \begin{Verbatim}[commandchars=\\\{\}]
{\color{incolor}In [{\color{incolor}1}]:} \PY{k}{def} \PY{n+nf}{fibonacci}\PY{p}{(}\PY{n}{n}\PY{p}{)}\PY{p}{:}
            \PY{l+s+s2}{\PYZdq{}}\PY{l+s+s2}{retourne le nombre de fibonacci pour l}\PY{l+s+s2}{\PYZsq{}}\PY{l+s+s2}{entier n}\PY{l+s+s2}{\PYZdq{}}
            \PY{c+c1}{\PYZsh{} pour les petites valeurs de n il n\PYZsq{}y a rien à calculer}
            \PY{k}{if} \PY{n}{n} \PY{o}{\PYZlt{}}\PY{o}{=} \PY{l+m+mi}{1}\PY{p}{:}
                \PY{k}{return} \PY{l+m+mi}{1}
            \PY{c+c1}{\PYZsh{} sinon on initialise f1 pour n\PYZhy{}1 et f2 pour n\PYZhy{}2}
            \PY{n}{f2}\PY{p}{,} \PY{n}{f1} \PY{o}{=} \PY{l+m+mi}{1}\PY{p}{,} \PY{l+m+mi}{1}
            \PY{c+c1}{\PYZsh{} et on itère n\PYZhy{}1 fois pour additionner}
            \PY{k}{for} \PY{n}{i} \PY{o+ow}{in} \PY{n+nb}{range}\PY{p}{(}\PY{l+m+mi}{2}\PY{p}{,} \PY{n}{n} \PY{o}{+} \PY{l+m+mi}{1}\PY{p}{)}\PY{p}{:}
                \PY{n}{f2}\PY{p}{,} \PY{n}{f1} \PY{o}{=} \PY{n}{f1}\PY{p}{,} \PY{n}{f1} \PY{o}{+} \PY{n}{f2}
        \PY{c+c1}{\PYZsh{}        print(i, f2, f1)}
            \PY{c+c1}{\PYZsh{} le résultat est dans f1}
            \PY{k}{return} \PY{n}{f1}
\end{Verbatim}


    Pour en faire un programme utilisable on va demander à l'utilisateur de
rentrer un nombre~; il faut le convertir en entier car \texttt{input}
renvoie une chaîne de caractères~:

    \begin{Verbatim}[commandchars=\\\{\}]
{\color{incolor}In [{\color{incolor}2}]:} \PY{n}{entier} \PY{o}{=} \PY{n+nb}{int}\PY{p}{(}\PY{n+nb}{input}\PY{p}{(}\PY{l+s+s2}{\PYZdq{}}\PY{l+s+s2}{Entrer un entier }\PY{l+s+s2}{\PYZdq{}}\PY{p}{)}\PY{p}{)}
\end{Verbatim}


    \begin{Verbatim}[commandchars=\\\{\}]
Entrer un entier 42

    \end{Verbatim}

    On imprime le résultat~:

    \begin{Verbatim}[commandchars=\\\{\}]
{\color{incolor}In [{\color{incolor}3}]:} \PY{n+nb}{print}\PY{p}{(}\PY{n}{f}\PY{l+s+s2}{\PYZdq{}}\PY{l+s+s2}{fibonacci(}\PY{l+s+si}{\PYZob{}entier\PYZcb{}}\PY{l+s+s2}{) = }\PY{l+s+s2}{\PYZob{}}\PY{l+s+s2}{fibonacci(entier)\PYZcb{}}\PY{l+s+s2}{\PYZdq{}}\PY{p}{)}
\end{Verbatim}


    \begin{Verbatim}[commandchars=\\\{\}]
fibonacci(42) = 433494437

    \end{Verbatim}

    \hypertarget{exercice}{%
\subsubsection{Exercice}\label{exercice}}

    Vous pouvez donc à présent~:

\begin{itemize}
\item
  exécuter le code dans ce notebook
\item
  télécharger ce code sur votre disque comme un fichier
  \texttt{fibonacci\_prompt.py}

  \begin{itemize}
  \tightlist
  \item
    utiliser pour cela le menu
    \emph{``File~-\textgreater{}~Download~as~-\textgreater{}~Python''}
  \item
    et \textbf{renommer le fichier obtenu} au besoin
  \end{itemize}
\item
  l'exécuter sous IDLE
\item
  le modifier, par exemple pour afficher les résultats intermédiaires

  \begin{itemize}
  \tightlist
  \item
    on a laissé exprès une fonction \texttt{print} en commentaire que
    vous pouvez réactiver simplement
  \end{itemize}
\item
  l'exécuter avec l'interpréteur Python comme ceci~:

  \texttt{\$\ python3\ fibonacci\_prompt.py}
\end{itemize}

    Ce code est volontairement simple et peu robuste pour ne pas l'alourdir.
Par exemple, ce programme se comporte mal si vous entrez un entier
négatif.\\

    Nous allons voir tout de suite une version légèrement différente qui va
vous permettre de donner la valeur d'entrée sur la ligne de commande.