    \hypertarget{obtenir-une-ruxe9ponse-de-lutilisateur}{%
\section{Obtenir une réponse de
l'utilisateur}\label{obtenir-une-ruxe9ponse-de-lutilisateur}}

    \hypertarget{compluxe9ment---niveau-basique}{%
\subsection{Complément - niveau
basique}\label{compluxe9ment---niveau-basique}}

    Occasionnellement, il peut être utile de poser une question à
l'utilisateur.

    \hypertarget{la-fonction-input}{%
\subsubsection{\texorpdfstring{La fonction
\texttt{input}}{La fonction input}}\label{la-fonction-input}}

    C'est le propos de la fonction \texttt{input}. Par exemple~:

    \begin{Verbatim}[commandchars=\\\{\}]
{\color{incolor}In [{\color{incolor}1}]:} \PY{n}{nom\PYZus{}ville} \PY{o}{=} \PY{n+nb}{input}\PY{p}{(}\PY{l+s+s2}{\PYZdq{}}\PY{l+s+s2}{Entrez le nom de la ville : }\PY{l+s+s2}{\PYZdq{}}\PY{p}{)}
        \PY{n+nb}{print}\PY{p}{(}\PY{n}{f}\PY{l+s+s2}{\PYZdq{}}\PY{l+s+s2}{nom\PYZus{}ville=}\PY{l+s+si}{\PYZob{}nom\PYZus{}ville\PYZcb{}}\PY{l+s+s2}{\PYZdq{}}\PY{p}{)}
\end{Verbatim}


    \begin{Verbatim}[commandchars=\\\{\}]
Entrez le nom de la ville : Metropolis
nom\_ville=Metropolis

    \end{Verbatim}

    \hypertarget{attention-uxe0-bien-vuxe9rifierconvertir}{%
\subsubsection{Attention à bien
vérifier/convertir}\label{attention-uxe0-bien-vuxe9rifierconvertir}}

    Notez bien que \texttt{input} renvoie \textbf{toujours une chaîne de
caractère} (\texttt{str}). C'est assez évident, mais il est très facile
de l'oublier et de passer cette chaîne directement à une fonction qui
s'attend à recevoir, par exemple, un nombre entier, auquel cas les
choses se passent mal~:

    \begin{Shaded}
\begin{Highlighting}[]
\OperatorTok{>>>} \BuiltInTok{input}\NormalTok{(}\StringTok{"nombre de lignes ? "}\NormalTok{) }\OperatorTok{+} \DecValTok{3}
\NormalTok{nombre de lignes ? }\DecValTok{12}
\NormalTok{Traceback (most recent call last):}
\NormalTok{  File }\StringTok{"<stdin>"}\NormalTok{, line }\DecValTok{1}\NormalTok{, }\KeywordTok{in} \OperatorTok{<}\NormalTok{module}\OperatorTok{>}
\PreprocessorTok{TypeError}\NormalTok{: must be }\BuiltInTok{str}\NormalTok{, }\KeywordTok{not} \BuiltInTok{int}
\end{Highlighting}
\end{Shaded}

    Dans ce cas il faut appeler la fonction \texttt{int} pour convertir le
résultat en un entier~:

    \begin{Verbatim}[commandchars=\\\{\}]
{\color{incolor}In [{\color{incolor}2}]:} \PY{n+nb}{int}\PY{p}{(}\PY{n+nb}{input}\PY{p}{(}\PY{l+s+s2}{\PYZdq{}}\PY{l+s+s2}{Nombre de lignes ? }\PY{l+s+s2}{\PYZdq{}}\PY{p}{)}\PY{p}{)} \PY{o}{+} \PY{l+m+mi}{3}
\end{Verbatim}


    \begin{Verbatim}[commandchars=\\\{\}]
Nombre de lignes ? 5

    \end{Verbatim}

\begin{Verbatim}[commandchars=\\\{\}]
{\color{outcolor}Out[{\color{outcolor}2}]:} 8
\end{Verbatim}
            
    \hypertarget{limitations}{%
\subsubsection{Limitations}\label{limitations}}

    Cette fonction peut être utile pour vos premiers pas en Python.\\

En pratique toutefois, on utilise assez peu cette fonction, car les
applications ``réelles'' viennent avec leur propre interface
utilisateur, souvent graphique, et disposent donc d'autres moyens que
celui-ci pour interagir avec l'utilisateur.\\

Les applications destinées à fonctionner dans un terminal, quant à
elles, reçoivent traditionnellement leurs données de la ligne de
commande. C'est le propos du module \texttt{argparse} que nous avons
déjà rencontré en première semaine.