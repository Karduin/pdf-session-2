    \hypertarget{tris-de-listes}{%
\section{Tris de listes}\label{tris-de-listes}}

    \hypertarget{compluxe9ment---niveau-basique}{%
\subsection{Complément - niveau
basique}\label{compluxe9ment---niveau-basique}}

    Python fournit une méthode standard pour trier une liste, qui s'appelle,
sans grande surprise, \texttt{sort}.

    \hypertarget{la-muxe9thode-sort}{%
\subsubsection{\texorpdfstring{La méthode
\texttt{sort}}{La méthode sort}}\label{la-muxe9thode-sort}}

    Voyons comment se comporte \texttt{sort} sur un exemple simple~:

    \begin{Verbatim}[commandchars=\\\{\}]
{\color{incolor}In [{\color{incolor}1}]:} \PY{n}{liste} \PY{o}{=} \PY{p}{[}\PY{l+m+mi}{8}\PY{p}{,} \PY{l+m+mi}{7}\PY{p}{,} \PY{l+m+mi}{4}\PY{p}{,} \PY{l+m+mi}{3}\PY{p}{,} \PY{l+m+mi}{2}\PY{p}{,} \PY{l+m+mi}{9}\PY{p}{,} \PY{l+m+mi}{1}\PY{p}{,} \PY{l+m+mi}{5}\PY{p}{,} \PY{l+m+mi}{6}\PY{p}{]}
        \PY{n+nb}{print}\PY{p}{(}\PY{l+s+s1}{\PYZsq{}}\PY{l+s+s1}{avant tri}\PY{l+s+s1}{\PYZsq{}}\PY{p}{,} \PY{n}{liste}\PY{p}{)}
        \PY{n}{liste}\PY{o}{.}\PY{n}{sort}\PY{p}{(}\PY{p}{)}
        \PY{n+nb}{print}\PY{p}{(}\PY{l+s+s1}{\PYZsq{}}\PY{l+s+s1}{apres tri}\PY{l+s+s1}{\PYZsq{}}\PY{p}{,} \PY{n}{liste}\PY{p}{)}
\end{Verbatim}


    \begin{Verbatim}[commandchars=\\\{\}]
avant tri [8, 7, 4, 3, 2, 9, 1, 5, 6]
apres tri [1, 2, 3, 4, 5, 6, 7, 8, 9]

    \end{Verbatim}

    On retrouve ici, avec l'instruction \texttt{liste.sort()} un cas d'appel
de méthode (ici \texttt{sort}) sur un objet (ici \texttt{liste}), comme
on l'avait vu dans la vidéo sur la notion d'objet.\\

    La première chose à remarquer est que la liste d'entrée a été modifiée,
on dit ``en place'', ou encore ``par effet de bord''. Voyons cela sous
pythontutor~:

    \begin{Verbatim}[commandchars=\\\{\}]
{\color{incolor}In [{\color{incolor} }]:} \PY{o}{\PYZpc{}}\PY{k}{load\PYZus{}ext} ipythontutor
\end{Verbatim}


    \begin{Verbatim}[commandchars=\\\{\}]
{\color{incolor}In [{\color{incolor} }]:} \PY{o}{\PYZpc{}\PYZpc{}}\PY{k}{ipythontutor} height=200 ratio=0.8
        liste = [3, 2, 9, 1]
        liste.sort()
\end{Verbatim}


    On aurait pu imaginer que la liste d'entrée soit restée inchangée, et
que la méthode de tri renvoie une copie triée de la liste, ce n'est pas
le choix qui a été fait, cela permet d'économiser des allocations
mémoire autant que possible et d'accélérer sensiblement le tri.

    \hypertarget{la-fonction-sorted}{%
\subsubsection{\texorpdfstring{La fonction
\texttt{sorted}}{La fonction sorted}}\label{la-fonction-sorted}}

    Si vous avez besoin de faire le tri sur une copie de votre liste, la
fonction \texttt{sorted} vous permet de le faire~:

    \begin{Verbatim}[commandchars=\\\{\}]
{\color{incolor}In [{\color{incolor} }]:} \PY{o}{\PYZpc{}\PYZpc{}}\PY{k}{ipythontutor} height=200 ratio=0.8
        liste1 = [3, 2, 9, 1]
        liste2 = sorted(liste1)
\end{Verbatim}


    \hypertarget{tri-duxe9croissant}{%
\subsubsection{Tri décroissant}\label{tri-duxe9croissant}}

    Revenons à la méthode \texttt{sort} et aux tris \emph{en place}. Par
défaut la liste est triée par ordre croissant, si au contraire vous
voulez l'ordre décroissant, faites comme ceci~:

    \begin{Verbatim}[commandchars=\\\{\}]
{\color{incolor}In [{\color{incolor}2}]:} \PY{n}{liste} \PY{o}{=} \PY{p}{[}\PY{l+m+mi}{8}\PY{p}{,} \PY{l+m+mi}{7}\PY{p}{,} \PY{l+m+mi}{4}\PY{p}{,} \PY{l+m+mi}{3}\PY{p}{,} \PY{l+m+mi}{2}\PY{p}{,} \PY{l+m+mi}{9}\PY{p}{,} \PY{l+m+mi}{1}\PY{p}{,} \PY{l+m+mi}{5}\PY{p}{,} \PY{l+m+mi}{6}\PY{p}{]}
        \PY{n+nb}{print}\PY{p}{(}\PY{l+s+s1}{\PYZsq{}}\PY{l+s+s1}{avant tri}\PY{l+s+s1}{\PYZsq{}}\PY{p}{,} \PY{n}{liste}\PY{p}{)}
        \PY{n}{liste}\PY{o}{.}\PY{n}{sort}\PY{p}{(}\PY{n}{reverse}\PY{o}{=}\PY{k+kc}{True}\PY{p}{)}
        \PY{n+nb}{print}\PY{p}{(}\PY{l+s+s1}{\PYZsq{}}\PY{l+s+s1}{apres tri décroissant}\PY{l+s+s1}{\PYZsq{}}\PY{p}{,} \PY{n}{liste}\PY{p}{)}
\end{Verbatim}


    \begin{Verbatim}[commandchars=\\\{\}]
avant tri [8, 7, 4, 3, 2, 9, 1, 5, 6]
apres tri décroissant [9, 8, 7, 6, 5, 4, 3, 2, 1]

    \end{Verbatim}

    Nous n'avons pas encore vu à quoi correspond cette formule
\texttt{reverse=True} dans l'appel à la méthode - ceci sera approfondi
dans le chapitre sur les appels de fonction - mais dans l'immédiat vous
pouvez utiliser cette technique telle quelle.

    \hypertarget{chauxeenes-de-caractuxe8res}{%
\subsubsection{Chaînes de
caractères}\label{chauxeenes-de-caractuxe8res}}

    Cette technique fonctionne très bien sur tous les types numériques
(enfin, à l'exception des complexes~; en guise d'exercice, pourquoi~?),
ainsi que sur les chaînes de caractères~:

    \begin{Verbatim}[commandchars=\\\{\}]
{\color{incolor}In [{\color{incolor}3}]:} \PY{n}{liste} \PY{o}{=} \PY{p}{[}\PY{l+s+s1}{\PYZsq{}}\PY{l+s+s1}{spam}\PY{l+s+s1}{\PYZsq{}}\PY{p}{,} \PY{l+s+s1}{\PYZsq{}}\PY{l+s+s1}{egg}\PY{l+s+s1}{\PYZsq{}}\PY{p}{,} \PY{l+s+s1}{\PYZsq{}}\PY{l+s+s1}{bacon}\PY{l+s+s1}{\PYZsq{}}\PY{p}{,} \PY{l+s+s1}{\PYZsq{}}\PY{l+s+s1}{beef}\PY{l+s+s1}{\PYZsq{}}\PY{p}{]}
        \PY{n}{liste}\PY{o}{.}\PY{n}{sort}\PY{p}{(}\PY{p}{)}
        \PY{n+nb}{print}\PY{p}{(}\PY{l+s+s1}{\PYZsq{}}\PY{l+s+s1}{après tri}\PY{l+s+s1}{\PYZsq{}}\PY{p}{,} \PY{n}{liste}\PY{p}{)}
\end{Verbatim}


    \begin{Verbatim}[commandchars=\\\{\}]
après tri ['bacon', 'beef', 'egg', 'spam']

    \end{Verbatim}

    Comme on s'y attend, il s'agit cette fois d'un \textbf{tri
lexicographique}, dérivé de l'ordre sur les caractères. Autrement dit,
c'est l'ordre du dictionnaire. Il faut souligner toutefois, pour les
personnes n'ayant jamais été exposées à l'informatique, que cet ordre,
quoique déterministe, est arbitraire en dehors des lettres de
l'alphabet.

    Ainsi par exemple~:

    \begin{Verbatim}[commandchars=\\\{\}]
{\color{incolor}In [{\color{incolor}4}]:} \PY{c+c1}{\PYZsh{} deux caractères minuscules se comparent}
        \PY{c+c1}{\PYZsh{} comme on s\PYZsq{}y attend}
        \PY{l+s+s1}{\PYZsq{}}\PY{l+s+s1}{a}\PY{l+s+s1}{\PYZsq{}} \PY{o}{\PYZlt{}} \PY{l+s+s1}{\PYZsq{}}\PY{l+s+s1}{z}\PY{l+s+s1}{\PYZsq{}}
\end{Verbatim}


\begin{Verbatim}[commandchars=\\\{\}]
{\color{outcolor}Out[{\color{outcolor}4}]:} True
\end{Verbatim}
            
    Bon, mais par contre~:

    \begin{Verbatim}[commandchars=\\\{\}]
{\color{incolor}In [{\color{incolor}5}]:} \PY{c+c1}{\PYZsh{} si l\PYZsq{}un est en minuscule et l\PYZsq{}autre en majuscule,}
        \PY{c+c1}{\PYZsh{} ce n\PYZsq{}est plus le cas}
        \PY{l+s+s1}{\PYZsq{}}\PY{l+s+s1}{Z}\PY{l+s+s1}{\PYZsq{}} \PY{o}{\PYZlt{}} \PY{l+s+s1}{\PYZsq{}}\PY{l+s+s1}{a}\PY{l+s+s1}{\PYZsq{}}
\end{Verbatim}


\begin{Verbatim}[commandchars=\\\{\}]
{\color{outcolor}Out[{\color{outcolor}5}]:} True
\end{Verbatim}
            
    Ce qui à son tour explique ceci~:

    \begin{Verbatim}[commandchars=\\\{\}]
{\color{incolor}In [{\color{incolor}6}]:} \PY{c+c1}{\PYZsh{} la conséquence de \PYZsq{}Z\PYZsq{} \PYZlt{} \PYZsq{}a\PYZsq{}, c\PYZsq{}est que}
        \PY{n}{liste} \PY{o}{=} \PY{p}{[}\PY{l+s+s1}{\PYZsq{}}\PY{l+s+s1}{abc}\PY{l+s+s1}{\PYZsq{}}\PY{p}{,} \PY{l+s+s1}{\PYZsq{}}\PY{l+s+s1}{Zoo}\PY{l+s+s1}{\PYZsq{}}\PY{p}{]}
        \PY{n}{liste}\PY{o}{.}\PY{n}{sort}\PY{p}{(}\PY{p}{)}
        \PY{n+nb}{print}\PY{p}{(}\PY{n}{liste}\PY{p}{)}
\end{Verbatim}


    \begin{Verbatim}[commandchars=\\\{\}]
['Zoo', 'abc']

    \end{Verbatim}

    Et lorsque les chaînes contiennent des espaces ou autres ponctuations,
le résultat du tri peut paraître surprenant~:

    \begin{Verbatim}[commandchars=\\\{\}]
{\color{incolor}In [{\color{incolor}7}]:} \PY{c+c1}{\PYZsh{} attention ici notre premiere chaîne commence par une espace}
        \PY{c+c1}{\PYZsh{} et le caractère \PYZsq{}Espace\PYZsq{} est plus petit}
        \PY{c+c1}{\PYZsh{} que tous les autres caractères imprimables}
        \PY{n}{liste} \PY{o}{=} \PY{p}{[}\PY{l+s+s1}{\PYZsq{}}\PY{l+s+s1}{ zoo}\PY{l+s+s1}{\PYZsq{}}\PY{p}{,} \PY{l+s+s1}{\PYZsq{}}\PY{l+s+s1}{ane}\PY{l+s+s1}{\PYZsq{}}\PY{p}{]}
        \PY{n}{liste}\PY{o}{.}\PY{n}{sort}\PY{p}{(}\PY{p}{)}
        \PY{n+nb}{print}\PY{p}{(}\PY{n}{liste}\PY{p}{)}
\end{Verbatim}


    \begin{Verbatim}[commandchars=\\\{\}]
[' zoo', 'ane']

    \end{Verbatim}

    \hypertarget{uxe0-suivre}{%
\subsubsection{À suivre}\label{uxe0-suivre}}

    Il est possible de définir soi-même le critère à utiliser pour trier une
liste, et nous verrons cela bientôt, une fois que nous aurons introduit
la notion de fonction.