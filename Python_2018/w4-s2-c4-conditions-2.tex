    \hypertarget{ruxe9capitulatif-sur-les-conditions-dans-un-if}{%
\section{\texorpdfstring{Récapitulatif sur les conditions dans un
\texttt{if}}{Récapitulatif sur les conditions dans un if}}\label{ruxe9capitulatif-sur-les-conditions-dans-un-if}}

    \hypertarget{compluxe9ment---niveau-basique}{%
\subsection{Complément - niveau
basique}\label{compluxe9ment---niveau-basique}}

    Dans ce complément nous résumons ce qu'il faut savoir pour écrire une
condition dans un \texttt{if}.

    \hypertarget{expression-vs-instruction}{%
\subsubsection{\texorpdfstring{Expression \emph{vs}
instruction}{Expression vs instruction}}\label{expression-vs-instruction}}

    Nous avons déjà introduit la différence entre instruction et expression,
lorsque nous avons vu l'expression conditionnelle~:

\begin{itemize}
	\item 
	une expression est un fragment de code qui ``retourne quelque chose'',
	item
	alors qu'une instruction permet bien souvent de faire une action, mais ne retourne
	rien.
\end{itemize}

    Ainsi parmi les notions que nous avons vues jusqu'ici, nous pouvons
citer dans un ordre arbitraire~:

\begin{longtable}[]{@{}rlr@{}}
\toprule
Instructions & & Expressions\tabularnewline
\midrule
\endhead
affectation & & appel de fonction\tabularnewline
\texttt{import} & & opérateurs \texttt{is}, \texttt{in}, \texttt{==},
\ldots{}\tabularnewline
instruction \texttt{if} & & expression conditionnelle\tabularnewline
instruction \texttt{for} & & compréhension(s)\tabularnewline
\bottomrule
\end{longtable}

    \hypertarget{toutes-les-expressions-sont-uxe9ligibles}{%
\subsubsection{Toutes les expressions sont
éligibles}\label{toutes-les-expressions-sont-uxe9ligibles}}

    Comme condition d'une instruction \texttt{if}, on peut mettre n'importe
quelle expression. On l'a déjà signalé, il n'est pas nécessaire que
cette expression retourne un booléen~:

    \begin{Verbatim}[commandchars=\\\{\}]
{\color{incolor}In [{\color{incolor}1}]:} \PY{c+c1}{\PYZsh{} dans ce code le test }
        \PY{c+c1}{\PYZsh{} if n \PYZpc{} 3:}
        \PY{c+c1}{\PYZsh{} est équivalent à}
        \PY{c+c1}{\PYZsh{} if n \PYZpc{} 3 != 0:}
        
        \PY{k}{for} \PY{n}{n} \PY{o+ow}{in} \PY{p}{(}\PY{l+m+mi}{18}\PY{p}{,} \PY{l+m+mi}{19}\PY{p}{)}\PY{p}{:}
            \PY{k}{if} \PY{n}{n} \PY{o}{\PYZpc{}} \PY{l+m+mi}{3}\PY{p}{:}
                \PY{n+nb}{print}\PY{p}{(}\PY{n}{f}\PY{l+s+s2}{\PYZdq{}}\PY{l+s+si}{\PYZob{}n\PYZcb{}}\PY{l+s+s2}{ non divisible par trois}\PY{l+s+s2}{\PYZdq{}}\PY{p}{)}
            \PY{k}{else}\PY{p}{:}
                \PY{n+nb}{print}\PY{p}{(}\PY{n}{f}\PY{l+s+s2}{\PYZdq{}}\PY{l+s+si}{\PYZob{}n\PYZcb{}}\PY{l+s+s2}{ divisible par trois}\PY{l+s+s2}{\PYZdq{}}\PY{p}{)}
\end{Verbatim}


    \begin{Verbatim}[commandchars=\\\{\}]
18 divisible par trois
19 non divisible par trois

    \end{Verbatim}

    \hypertarget{une-valeur-est-elle-vraie}{%
\subsubsection{Une valeur est-elle ``vraie''
?}\label{une-valeur-est-elle-vraie}}

    Se pose dès lors la question de savoir précisément quelles valeurs sont
considérées comme \emph{vraies} par l'instruction \texttt{if}.\\

Parmi les types de base, nous avons déjà eu l'occasion de l'évoquer, les
valeurs \emph{fausses} sont typiquement~:

\begin{itemize}
	\item 
	0 pour les valeurs numériques~;
	\item
	les objets vides pour les chaînes, listes, ensembles,
	dictionnaires, etc.
\end{itemize}

    Pour en avoir le cœur net, pensez à utiliser dans le terminal interactif
la fonction \texttt{bool}. Comme pour toutes les fonctions qui portent
le nom d'un type, la fonction \texttt{bool} est un constructeur qui
fabrique un objet booléen.\\

Si vous appelez \texttt{bool} sur un objet, la valeur de retour - qui
est donc par construction une valeur booléenne - vous indique, cette
fois sans ambiguïté - comment se comportera \texttt{if} avec cette
entrée.

    \begin{Verbatim}[commandchars=\\\{\}]
{\color{incolor}In [{\color{incolor}2}]:} \PY{k}{def} \PY{n+nf}{show\PYZus{}bool}\PY{p}{(}\PY{n}{x}\PY{p}{)}\PY{p}{:}
            \PY{n+nb}{print}\PY{p}{(}\PY{n}{f}\PY{l+s+s2}{\PYZdq{}}\PY{l+s+s2}{condition }\PY{l+s+s2}{\PYZob{}}\PY{l+s+s2}{repr(x):\PYZgt{}10\PYZcb{} considérée comme }\PY{l+s+s2}{\PYZob{}}\PY{l+s+s2}{bool(x)\PYZcb{}}\PY{l+s+s2}{\PYZdq{}}\PY{p}{)}
\end{Verbatim}


    \begin{Verbatim}[commandchars=\\\{\}]
{\color{incolor}In [{\color{incolor}3}]:} \PY{k}{for} \PY{n}{exp} \PY{o+ow}{in} \PY{p}{[}\PY{k+kc}{None}\PY{p}{,} \PY{l+s+s2}{\PYZdq{}}\PY{l+s+s2}{\PYZdq{}}\PY{p}{,} \PY{l+s+s1}{\PYZsq{}}\PY{l+s+s1}{a}\PY{l+s+s1}{\PYZsq{}}\PY{p}{,} \PY{p}{[}\PY{p}{]}\PY{p}{,} \PY{p}{[}\PY{l+m+mi}{1}\PY{p}{]}\PY{p}{,} \PY{p}{(}\PY{p}{)}\PY{p}{,} \PY{p}{(}\PY{l+m+mi}{1}\PY{p}{,} \PY{l+m+mi}{2}\PY{p}{)}\PY{p}{,} \PY{p}{\PYZob{}}\PY{p}{\PYZcb{}}\PY{p}{,} \PY{p}{\PYZob{}}\PY{l+s+s1}{\PYZsq{}}\PY{l+s+s1}{a}\PY{l+s+s1}{\PYZsq{}}\PY{p}{:} \PY{l+m+mi}{1}\PY{p}{\PYZcb{}}\PY{p}{,} \PY{n+nb}{set}\PY{p}{(}\PY{p}{)}\PY{p}{,} \PY{p}{\PYZob{}}\PY{l+m+mi}{1}\PY{p}{\PYZcb{}}\PY{p}{]}\PY{p}{:}
            \PY{n}{show\PYZus{}bool}\PY{p}{(}\PY{n}{exp}\PY{p}{)}
\end{Verbatim}


    \begin{Verbatim}[commandchars=\\\{\}]
condition       None considérée comme False
condition         '' considérée comme False
condition        'a' considérée comme True
condition         [] considérée comme False
condition        [1] considérée comme True
condition         () considérée comme False
condition     (1, 2) considérée comme True
condition         \{\} considérée comme False
condition   \{'a': 1\} considérée comme True
condition      set() considérée comme False
condition        \{1\} considérée comme True

    \end{Verbatim}

    \hypertarget{quelques-exemples-dexpressions}{%
\subsubsection{Quelques exemples
d'expressions}\label{quelques-exemples-dexpressions}}

    \hypertarget{ruxe9fuxe9rence-uxe0-une-variable-et-duxe9rivuxe9s}{%
\subparagraph{Référence à une variable et
dérivés}\label{ruxe9fuxe9rence-uxe0-une-variable-et-duxe9rivuxe9s}}

    \begin{Verbatim}[commandchars=\\\{\}]
{\color{incolor}In [{\color{incolor}4}]:} \PY{n}{a} \PY{o}{=} \PY{n+nb}{list}\PY{p}{(}\PY{n+nb}{range}\PY{p}{(}\PY{l+m+mi}{4}\PY{p}{)}\PY{p}{)}
        \PY{n+nb}{print}\PY{p}{(}\PY{n}{a}\PY{p}{)}
\end{Verbatim}


    \begin{Verbatim}[commandchars=\\\{\}]
[0, 1, 2, 3]

    \end{Verbatim}

    \begin{Verbatim}[commandchars=\\\{\}]
{\color{incolor}In [{\color{incolor}5}]:} \PY{k}{if} \PY{n}{a}\PY{p}{:}
            \PY{n+nb}{print}\PY{p}{(}\PY{l+s+s2}{\PYZdq{}}\PY{l+s+s2}{a n}\PY{l+s+s2}{\PYZsq{}}\PY{l+s+s2}{est pas vide}\PY{l+s+s2}{\PYZdq{}}\PY{p}{)}
        \PY{k}{if} \PY{n}{a}\PY{p}{[}\PY{l+m+mi}{0}\PY{p}{]}\PY{p}{:}
            \PY{n+nb}{print}\PY{p}{(}\PY{l+s+s2}{\PYZdq{}}\PY{l+s+s2}{on ne passe pas par ici}\PY{l+s+s2}{\PYZdq{}}\PY{p}{)}
        \PY{k}{if} \PY{n}{a}\PY{p}{[}\PY{l+m+mi}{1}\PY{p}{]}\PY{p}{:}
            \PY{n+nb}{print}\PY{p}{(}\PY{l+s+s2}{\PYZdq{}}\PY{l+s+s2}{a[1] n}\PY{l+s+s2}{\PYZsq{}}\PY{l+s+s2}{est pas nul}\PY{l+s+s2}{\PYZdq{}}\PY{p}{)}
\end{Verbatim}


    \begin{Verbatim}[commandchars=\\\{\}]
a n'est pas vide
a[1] n'est pas nul

    \end{Verbatim}

    \hypertarget{appels-de-fonction-ou-de-muxe9thode}{%
\subparagraph{Appels de fonction ou de
méthode}\label{appels-de-fonction-ou-de-muxe9thode}}

    \begin{Verbatim}[commandchars=\\\{\}]
{\color{incolor}In [{\color{incolor}6}]:} \PY{n}{chaine} \PY{o}{=} \PY{l+s+s2}{\PYZdq{}}\PY{l+s+s2}{jean}\PY{l+s+s2}{\PYZdq{}}
        \PY{k}{if} \PY{n}{chaine}\PY{o}{.}\PY{n}{upper}\PY{p}{(}\PY{p}{)}\PY{p}{:}
            \PY{n+nb}{print}\PY{p}{(}\PY{l+s+s2}{\PYZdq{}}\PY{l+s+s2}{la chaine mise en majuscule n}\PY{l+s+s2}{\PYZsq{}}\PY{l+s+s2}{est pas vide}\PY{l+s+s2}{\PYZdq{}}\PY{p}{)}
\end{Verbatim}


    \begin{Verbatim}[commandchars=\\\{\}]
la chaine mise en majuscule n'est pas vide

    \end{Verbatim}

    \begin{Verbatim}[commandchars=\\\{\}]
{\color{incolor}In [{\color{incolor}7}]:} \PY{c+c1}{\PYZsh{} on rappelle qu\PYZsq{}une fonction qui ne fait pas \PYZsq{}return\PYZsq{} retourne None}
        \PY{k}{def} \PY{n+nf}{procedure}\PY{p}{(}\PY{n}{a}\PY{p}{,} \PY{n}{b}\PY{p}{,} \PY{n}{c}\PY{p}{)}\PY{p}{:}
            \PY{l+s+s2}{\PYZdq{}}\PY{l+s+s2}{cette fonction ne retourne rien}\PY{l+s+s2}{\PYZdq{}}
            \PY{k}{pass}
        
        \PY{k}{if} \PY{n}{procedure}\PY{p}{(}\PY{l+m+mi}{1}\PY{p}{,} \PY{l+m+mi}{2}\PY{p}{,} \PY{l+m+mi}{3}\PY{p}{)}\PY{p}{:}
            \PY{n+nb}{print}\PY{p}{(}\PY{l+s+s2}{\PYZdq{}}\PY{l+s+s2}{ne passe pas ici car procedure retourne None}\PY{l+s+s2}{\PYZdq{}}\PY{p}{)}
        \PY{k}{else}\PY{p}{:}
            \PY{n+nb}{print}\PY{p}{(}\PY{l+s+s2}{\PYZdq{}}\PY{l+s+s2}{par contre on passe ici}\PY{l+s+s2}{\PYZdq{}}\PY{p}{)}
\end{Verbatim}


    \begin{Verbatim}[commandchars=\\\{\}]
par contre on passe ici

    \end{Verbatim}

    \hypertarget{compruxe9hensions}{%
\subparagraph{Compréhensions\\\\}\label{compruxe9hensions}}

    Il découle de ce qui précède qu'on peut tout à fait mettre une
compréhension comme condition, ce qui peut être utile pour savoir si au
moins un élément remplit une condition, comme par exemple~:

    \begin{Verbatim}[commandchars=\\\{\}]
{\color{incolor}In [{\color{incolor}8}]:} \PY{n}{inputs} \PY{o}{=} \PY{p}{[}\PY{l+m+mi}{23}\PY{p}{,} \PY{l+m+mi}{65}\PY{p}{,} \PY{l+m+mi}{24}\PY{p}{]}
        
        \PY{c+c1}{\PYZsh{} y a\PYZhy{}t\PYZhy{}il dans inputs au moins un nombre }
        \PY{c+c1}{\PYZsh{} dont le carré est de la forme 10*n+5}
        \PY{k}{def} \PY{n+nf}{condition}\PY{p}{(}\PY{n}{n}\PY{p}{)}\PY{p}{:}
            \PY{k}{return} \PY{p}{(}\PY{n}{n} \PY{o}{*} \PY{n}{n}\PY{p}{)} \PY{o}{\PYZpc{}} \PY{l+m+mi}{10} \PY{o}{==} \PY{l+m+mi}{5}
        
        \PY{k}{if} \PY{p}{[}\PY{n}{value} \PY{k}{for} \PY{n}{value} \PY{o+ow}{in} \PY{n}{inputs} \PY{k}{if} \PY{n}{condition}\PY{p}{(}\PY{n}{value}\PY{p}{)}\PY{p}{]}\PY{p}{:}
            \PY{n+nb}{print}\PY{p}{(}\PY{l+s+s2}{\PYZdq{}}\PY{l+s+s2}{au moins une entrée convient}\PY{l+s+s2}{\PYZdq{}}\PY{p}{)}
\end{Verbatim}


    \begin{Verbatim}[commandchars=\\\{\}]
au moins une entrée convient

    \end{Verbatim}

    \hypertarget{opuxe9rateurs}{%
\subparagraph{Opérateurs\\\\}\label{opuxe9rateurs}}

    Nous avons déjà eu l'occasion de rencontrer la plupart des opérateurs de
comparaison du langage, dont voici à nouveau les principaux~:

\begin{longtable}[]{@{}rlr@{}}
\toprule
Famille & & Exemples\tabularnewline
\midrule
\endhead
Égalité & & \texttt{==}, \texttt{!=}, \texttt{is},
\texttt{is\ not}\tabularnewline
Appartenance & & \texttt{in}\tabularnewline
Comparaison & & \texttt{\textless{}=}, \texttt{\textless{}},
\texttt{\textgreater{}}, \texttt{\textgreater{}=}\tabularnewline
Logiques & & \texttt{and}, \texttt{or}, \texttt{not}\tabularnewline
\bottomrule
\end{longtable}

    \hypertarget{compluxe9ment---niveau-intermuxe9diaire}{%
\subsection{Complément - niveau
intermédiaire}\label{compluxe9ment---niveau-intermuxe9diaire}}

    \hypertarget{remarques-sur-les-opuxe9rateurs}{%
\subsubsection{Remarques sur les
opérateurs}\label{remarques-sur-les-opuxe9rateurs}}

    Voici enfin quelques remarques sur ces opérateurs

    \hypertarget{opuxe9rateur-duxe9galituxe9}{%
\subparagraph{\texorpdfstring{opérateur d'égalité
\texttt{==}\\\\}{opérateur d'égalité ==}}\label{opuxe9rateur-duxe9galituxe9}}

    L'opérateur \texttt{==} ne fonctionne en général (sauf pour les nombres)
que sur des objets de même type~; c'est-à-dire que notamment un tuple ne
sera jamais égal à une liste~:

    \begin{Verbatim}[commandchars=\\\{\}]
{\color{incolor}In [{\color{incolor}9}]:} \PY{p}{[}\PY{p}{]} \PY{o}{==} \PY{p}{(}\PY{p}{)}
\end{Verbatim}


\begin{Verbatim}[commandchars=\\\{\}]
{\color{outcolor}Out[{\color{outcolor}9}]:} False
\end{Verbatim}
            
    \begin{Verbatim}[commandchars=\\\{\}]
{\color{incolor}In [{\color{incolor}10}]:} \PY{p}{[}\PY{l+m+mi}{1}\PY{p}{,} \PY{l+m+mi}{2}\PY{p}{]} \PY{o}{==} \PY{p}{(}\PY{l+m+mi}{1}\PY{p}{,} \PY{l+m+mi}{2}\PY{p}{)}
\end{Verbatim}


\begin{Verbatim}[commandchars=\\\{\}]
{\color{outcolor}Out[{\color{outcolor}10}]:} False
\end{Verbatim}
            
    \hypertarget{opuxe9rateur-logiques}{%
\subparagraph{opérateur logiques\\\\}\label{opuxe9rateur-logiques}}

    Comme c'est le cas avec par exemple les opérateurs arithmétiques, les
opérateurs logiques ont une \emph{priorité}, qui précise le sens des
phrases non parenthésées. C'est-à-dire pour être explicite, que de la
même manière que

\begin{verbatim}
12 + 4 * 8 
\end{verbatim}

est équivalent à

\begin{verbatim}
12 + ( 4 * 8 )
\end{verbatim}

pour les booléens il existe une règle de ce genre et

\begin{verbatim}
a and not b or c and d
\end{verbatim}

est équivalent à

\begin{verbatim}
(a and (not b)) or (c and d)
\end{verbatim}

Mais en fait, il est assez facile de s'emmêler dans ces priorités, et
c'est pourquoi il est \textbf{très fortement conseillé} de parenthéser.

    \hypertarget{opuxe9rateurs-logiques-2}{%
\subparagraph{opérateurs logiques (2)\\\\}\label{opuxe9rateurs-logiques-2}}

    Remarquez aussi que les opérateurs logiques peuvent être appliqués à des
valeurs qui ne sont pas booléennes~:

    \begin{Verbatim}[commandchars=\\\{\}]
{\color{incolor}In [{\color{incolor}11}]:} \PY{l+m+mi}{2} \PY{o+ow}{and} \PY{p}{[}\PY{l+m+mi}{1}\PY{p}{,} \PY{l+m+mi}{2}\PY{p}{]}
\end{Verbatim}


\begin{Verbatim}[commandchars=\\\{\}]
{\color{outcolor}Out[{\color{outcolor}11}]:} [1, 2]
\end{Verbatim}
            
    \begin{Verbatim}[commandchars=\\\{\}]
{\color{incolor}In [{\color{incolor}12}]:} \PY{k+kc}{None} \PY{o+ow}{or} \PY{l+s+s2}{\PYZdq{}}\PY{l+s+s2}{abcde}\PY{l+s+s2}{\PYZdq{}}
\end{Verbatim}


\begin{Verbatim}[commandchars=\\\{\}]
{\color{outcolor}Out[{\color{outcolor}12}]:} 'abcde'
\end{Verbatim}
            
    Dans la logique de l'évaluation paresseuse qu'on a vue récemment,
remarquez que lorsque l'évaluation d'un \texttt{and} ou d'un \texttt{or}
ne peut pas être court-circuitée, le résultat est alors toujours le
résultat de la dernière expression évaluée~:

    \begin{Verbatim}[commandchars=\\\{\}]
{\color{incolor}In [{\color{incolor}13}]:} \PY{l+m+mi}{1} \PY{o+ow}{and} \PY{l+m+mi}{2} \PY{o+ow}{and} \PY{l+m+mi}{3}
\end{Verbatim}


\begin{Verbatim}[commandchars=\\\{\}]
{\color{outcolor}Out[{\color{outcolor}13}]:} 3
\end{Verbatim}
            
    \begin{Verbatim}[commandchars=\\\{\}]
{\color{incolor}In [{\color{incolor}14}]:} \PY{l+m+mi}{1} \PY{o+ow}{and} \PY{l+m+mi}{2} \PY{o+ow}{and} \PY{l+m+mi}{3} \PY{o+ow}{and} \PY{l+s+s1}{\PYZsq{}}\PY{l+s+s1}{\PYZsq{}} \PY{o+ow}{and} \PY{l+m+mi}{4}
\end{Verbatim}


\begin{Verbatim}[commandchars=\\\{\}]
{\color{outcolor}Out[{\color{outcolor}14}]:} ''
\end{Verbatim}
            
    \begin{Verbatim}[commandchars=\\\{\}]
{\color{incolor}In [{\color{incolor}15}]:} \PY{p}{[}\PY{p}{]} \PY{o+ow}{or} \PY{l+s+s2}{\PYZdq{}}\PY{l+s+s2}{\PYZdq{}} \PY{o+ow}{or} \PY{p}{\PYZob{}}\PY{p}{\PYZcb{}}
\end{Verbatim}


\begin{Verbatim}[commandchars=\\\{\}]
{\color{outcolor}Out[{\color{outcolor}15}]:} \{\}
\end{Verbatim}
            
    \begin{Verbatim}[commandchars=\\\{\}]
{\color{incolor}In [{\color{incolor}16}]:} \PY{p}{[}\PY{p}{]} \PY{o+ow}{or} \PY{l+s+s2}{\PYZdq{}}\PY{l+s+s2}{\PYZdq{}} \PY{o+ow}{or} \PY{p}{\PYZob{}}\PY{p}{\PYZcb{}} \PY{o+ow}{or} \PY{l+m+mi}{4} \PY{o+ow}{or} \PY{n+nb}{set}\PY{p}{(}\PY{p}{)}
\end{Verbatim}


\begin{Verbatim}[commandchars=\\\{\}]
{\color{outcolor}Out[{\color{outcolor}16}]:} 4
\end{Verbatim}
            
    \hypertarget{expression-conditionnelle-dans-une-instruction-if}{%
\subsubsection{\texorpdfstring{Expression conditionnelle dans une
instruction
\texttt{if}}{Expression conditionnelle dans une instruction if}}\label{expression-conditionnelle-dans-une-instruction-if}}

    En toute rigueur on peut aussi mettre un
\texttt{\textless{}\textgreater{}\ if\ \textless{}\textgreater{}\ else\ \textless{}\textgreater{}}
- donc une expression conditionnelle - comme condition dans une
instruction \texttt{if}. Nous le signalons pour bien illustrer la
logique du langage, mais cette pratique n'est bien sûr pas du tout
conseillée.

    \begin{Verbatim}[commandchars=\\\{\}]
{\color{incolor}In [{\color{incolor}17}]:} \PY{c+c1}{\PYZsh{} cet exemple est volontairement tiré par les cheveux }
         \PY{c+c1}{\PYZsh{} pour bien montrer qu\PYZsq{}on peut mettre n\PYZsq{}importe quelle expression comme condition}
         \PY{n}{a} \PY{o}{=} \PY{l+m+mi}{1}
         \PY{c+c1}{\PYZsh{} ceci est franchement illisible}
         \PY{k}{if} \PY{l+m+mi}{0} \PY{k}{if} \PY{o+ow}{not} \PY{n}{a} \PY{k}{else} \PY{l+m+mi}{2}\PY{p}{:}
             \PY{n+nb}{print}\PY{p}{(}\PY{l+s+s2}{\PYZdq{}}\PY{l+s+s2}{une construction illisible}\PY{l+s+s2}{\PYZdq{}}\PY{p}{)}
         \PY{c+c1}{\PYZsh{} et encore pire}
         \PY{k}{if} \PY{l+m+mi}{0} \PY{k}{if} \PY{n}{a} \PY{k}{else} \PY{l+m+mi}{3} \PY{k}{if} \PY{n}{a} \PY{o}{+} \PY{l+m+mi}{1} \PY{k}{else} \PY{l+m+mi}{2}\PY{p}{:}
             \PY{n+nb}{print}\PY{p}{(}\PY{l+s+s2}{\PYZdq{}}\PY{l+s+s2}{encore pire}\PY{l+s+s2}{\PYZdq{}}\PY{p}{)}
\end{Verbatim}


    \begin{Verbatim}[commandchars=\\\{\}]
une construction illisible

    \end{Verbatim}

    \hypertarget{pour-en-savoir-plus}{%
\subsubsection{Pour en savoir plus}\label{pour-en-savoir-plus}}

    \href{https://docs.python.org/3/tutorial/datastructures.html\#more-on-conditions}{https://docs.python.org/3/tutorial/datastructures.html\#more-on-conditions}

    \hypertarget{types-duxe9finis-par-lutilisateur}{%
\subsubsection{Types définis par
l'utilisateur}\label{types-duxe9finis-par-lutilisateur}}

    Pour anticiper un tout petit peu, nous verrons que les classes en Python
vous donnent le moyen de définir vos propres types d'objets. Nous
verrons à cette occasion qu'il est possible d'indiquer à python quels
sont les objets de type \texttt{MaClasse} qui doivent être considérés
comme \texttt{True} ou comme \texttt{False}.\\

De manière plus générale, tous les traits natifs du langage sont
redéfinissables sur les classes. Nous verrons par exemple également
comment donner du sens à des phrases comme

\begin{verbatim}
mon_objet = MaClasse()
if mon_objet:
    <faire quelque chose>
\end{verbatim}

ou encore

\begin{verbatim}
mon_objet = MaClasse()
for partie in mon_objet:
    <faire quelque chose sur partie>
\end{verbatim}

Mais n'anticipons pas trop, rendez-vous en semaine 6.